\documentclass[../document.tex]{subfiles}
\begin{document}
\section{Results}
In the section we show and evaluate the model's results. We will start by presenting to the reader the networks score on each metric, then we will use the best model to predict the traversability in real world terrains. 
\subsection{Experiment Setup}
\subsubsection{Hardware}
We run all the experiment on a Ubuntu 18.10  work station equipped with a Ryzen 2700x, a powerful CPU with 8 cores and 16 threads, and a NVIDIA 1080 GPU with 8GB of dedicated RAM.
\subsubsection{Dataset}
To perform classification, we select a threshold of $0.2$m on a time window, $\Delta t$, of two seconds to label the patches, meaning that a patch with an advancement less than $20$ centimeters is labeled as \emph{no traversable} and viceversa. The  This processed is explained in detail in the previous chapter. While for the regression, we did not label the patch and directly regress on the advancement.

Initially, to train the models we first use Standard Gradient Descent with momentum set to $0.95$ and weight decay to $1e-4$ with an initial learning rate of $1e-3$ as was originally proposed to train residual network \cite{he2015deep}. However, we later utilize Leslie Smith's 1cycle policy \cite{1cycle} that allows us to trian the network faster and with an higher accuracy. We minimise the binary Cross Entropy for the classifier and the  Mean Square Error (MSE) for the regressor.
\subsubsection{Experimental validation}
We select as \emph{validation} set ten percent of the training data. Since we store each run of Krock as a \emph{.csv} file, validation and train set do not overlap. 
The test set is composed entirely by the Quarry map, a real world scenario. Table \ref{table: maps} tells in detail the configuration used in each map of all sets.

We also evaluate the model on the following additonal maps
\todo[inline]{ add arck rocks}
\subsubsection{Metrics}

\paragraph{Classification:} To evaluate the model's classification performance we used two metrics: \emph{accuracy} and \emph{AUC-ROC Curve}. Accuracy scores the number of correct predictions made by the network while AUC-ROC Curve represents degree or measure of separability, informally it tells how much model is capable of distinguishing between classes. For each experiment, we select the model with the higher AUC-ROC Curve during training to be evaluated on the test set.



\paragraph{Regression:} We used the Mean Square Error to evaluate the model's performance.
\subsection{Quantitative Results}
\subsubsection{Model selection}
We compared two different \emph{micro-resnet} and the \emph{vanilla} cnn from the previous Chapter. We evaluate those models using a time window of two second, a threshold of $20$cm and the data augmentation techniques described before. We run five experiments per architecture and we select the best performing network, the results are showed in the following table. 

\begin{table*}[h]
  \centering
  \ra{1.2}
  \begin{tabular}{@{}lcccc@{}}
  \toprule
   && Vanilla & \multicolumn{2}{c}{MicroResnetSE} \\
  \cline{3-5}
  && & $3\times 3$ stride $1$ & $7\times7$ stride $2$\\ 
  \cline{3-5}
  \multirow{2}{*}{AUC} & Top & 0.892 & 0.888 & \textbf{0.896}\\
   & Mean & \textbf{0.890} & 0.883 & 0.888\\
  \cline{1-5}
  Params & & 974,351 & 313,642 & 314,282  \\
  \bottomrule   
\end{tabular}
\caption{Model comparison on the test set.}
\end{table*}
\todo[inline]{Luca told me is better to split the models like Model1 and Model2 etc}

Based on this data We select \emph{micro-resnet} with squeeze and excitation and a starting convolution's kernel size of $7\times7$ with stride of $2$. This model has one third of the parameter of the origal model proposed by Chavez-Garcia et all \cite{omar2018traversability}. 

As proof of work, we also train the best network architecture, MicroResnetSE with a first convolution's kernel size of $7 \times 7$ and stride$=2$, with and without the Squeeze and Excitation operator.
\begin{table*}[h]
  \centering
  \ra{1.2}
  \begin{tabular}{@{}lccc@{}}
  \toprule
  &  MicroResnet$7\times7$ & MicroResnet$7\times7$-SE  & Improvement \\
  \cline{1-4}
   Top & 0.875 & \textbf{0.896} & $+0.021$ \\
   Mean & 0.867 & \textbf{0.888} & $+0.021$ \\
  \bottomrule   
\end{tabular}
\caption{AUC top value and mean value for MicroResnet with a fist convolution of $7\times7$ and stride $=2$ with and without the SE module. The improvement is the same.}
\end{table*}

\subsection{Final results}
The following table shows in deep the score of the best network for each dataset.
\begin{table*}[h]
    \centering
    \ra{1.2}
    \begin{tabular}{@{}llccccc@{}}
    \toprule
    % \multicolumn{8}{c}{Quantitative evaluation in simulation} \\
    \multicolumn{2}{c}{Dataset} && \multicolumn{2}{c}{micro-resnet} & Size & Resolution(cm/px) \\
    \cmidrule{1-2} \cmidrule{4-5}
    Type     &  Name  & Samples & ACC  &  AUC    & & \\
    \toprule
      \multirow{3}{*}{Synthetic}  & Training   & 429312 & - & - & & 2\\
      &  Validation   & 44032 &  95.2 \% &  0.961 & & 2 \\
      & Arc Rocks & 37273 &  85.5 \% &  0.888 & & 2 \\
      \cmidrule{2-7}
    \multirow{3}{*}{\makecell[l]{Real\\evaluation}} & Quarry & 36224 &  88.2 \%&  0.896& & 2\\
    & foo & TODO & & & & \\
    & baaa & TODO & & & & \\
    \bottomrule   
\end{tabular}
\caption{Final results on different datasets.}
\end{table*}
\todo[inline]{I have actually never talk about surf rocks}
Moreover, we would like to also show the different steps we made to reach this result. The following table shows the metric's score without any data-augmentation.
\todo[inline]{add result with and without data agu}
Adding dropout increases the results.
\todo[inline]{table with results}
With dropout plus coarse dropout.
\todo[inline]{table with results}
\subsection{Qualitative results}
We qualitative evaluate the model predictions by showing the traversability probability on different maps in 3D as a texture. Specifically, we used a sliding window to extract the patches from the heightmaps and create a texture based on the model's output for the traversable class. We then apply this texture on the image and colour using a colormap. For each map we show the traversability from bottom to top, top to bottom, left to right and right to left since those are the most human understandable.
\subsubsection{Quarry}
% \todo[inline]{add quarry textures from bottom to top}

The first map we evaluate is Quarry. This maps has some interesting features, such as the three bis slopes and the rocky ground on top. We expect the trail on the slopes to be traversable at almost any rotation, expecially from left to right and viceversa. While, the top part should be not traversable in any case. The following figure shows the traversability probability on the map.
\begin{figure}[H]
\centering
\begin{subfigure}[b]{0.45\textwidth}
  \includegraphics[width=\linewidth]{../img/4/traversability/quarry/270.png}  
\end{subfigure}
\begin{subfigure}[b]{0.45\textwidth}
    \includegraphics[width=\linewidth]{../img/4/traversability/quarry/0.png}  
\end{subfigure}
\begin{subfigure}[b]{0.45\textwidth}
  \includegraphics[width=\linewidth]{../img/4/traversability/quarry/90.png}  
\end{subfigure}
\begin{subfigure}[b]{0.45\textwidth}
    \includegraphics[width=\linewidth]{../img/4/traversability/quarry/180.png}  
\end{subfigure}

\caption{Traversability probability on the Quarry map for different Krock's rotation. The values are obstained by sliding a window on the map to create the patches to get the predictions from the model.}
\end{figure}
\todo[inline]{need to add arrows}
We can start our discussion by commenting the first image, Krock moving from bottom to top. Correctly, all the flat regions were label with high confidence as traversable, expecially the lower part of the map. There are two interesting regions, the bumps on the bottom right corner and the slopes. The following images higlights this two details
\todo[inline]{add images}
The slopes traversability is depends on the orientation. When the robot is moving up-hill, the propability is low while when it is going down those patches are completely traversable. Also, we the robot is moving from left to right and right to left it is able to traverse the slopes since the inclination is not too big to make the robot fall. 

For completeness, more images follow with their traversability probability.
\todo[inline]{add images of bars, arcs rocks}

\end{document}