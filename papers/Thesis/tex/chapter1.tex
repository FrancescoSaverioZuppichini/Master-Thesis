\documentclass[../document.tex]{subfiles}
\begin{document}
\chapter{Related Work}
The learning and perception of traversability is
a fundamental competence for both organisms and autonomous mobile robots since most of their
actions depend on their mobility \cite{ugur2010traversability}. 
Visual perception is known to be used in most all animals to correctly estimate if an environment can be traversed or not.
Similar, a wide array of autonomous robots adopt local sensors to mimic the visual properties of animals to extract geometric information of the surrounding and plan a safe path through it. 

Different methodologies have been proposed to collect the data and then learn to correctly navigate the environment.
Most of the methodologies rely on supervised learning, where first the data is gathered and then a machine learning algorithm is trained sample to correctly predict the traversability of those samples.
Among the huge numbers of methods proposed, there are two categories based on the input data: geometric and appearance based methods. 

Geometric methods aim to detect traversability using geometric properties of surfaces such as distances in space and shapes. Those properties are usually slopes, bumps, and ramps. Since nearly the entire world has been surveyed at 1 m accuracy \cite{sofman2006improving}, 
outdoor robot navigation can benefit from the availability of overhead imagery. For this reason, elevation data has also been used to extract geometric information. Chavez-Garcia et al. \cite{omar2018traversability}, proposed a framework to estimate traversability using only elevation data in the form of height maps. 

Elevation data can also be estimated by flying drones. Delmerico et al. \cite{delmerico2016active} proposed a collaborative search and rescue system in which a flying robot that explores the map and creates an elevation map to guide the ground robot to the goal. They train on the fly a convolutional neural network to segment the terrain in different traversable classes.  

Whereas appearance methods, to a greater extent related to camera images processing and cognitive analyses, have the objective of recognising colours and patterns not related to the common appearance of terrains, such as grass, rocks or vegetation. Those images can be used to directly estimate the traversability cost. 

Historically, the collected data is first preprocessed to extract texture features that are used to fit a classic machine learning classified such us an SVM \cite{ugur2010traversability} or Gaussian models \cite{sofman2006improving}. Those techniques rely on texture descriptors, for example, Local Binary Pattern \cite{ojala2002multiresolution}, to extract features from the raw data images obtained from local sensors such as cameras.
With the rise of deep learning methods in computer vision, deep convolution neural network has been trained directly on the raw RGB images bypassing the need to define characteristic features.

One recent example is the work of Giusti et al. \cite{giusti2015amachine} where a deep neural network was training on real-world hiking data collected using head-mounted cameras to teach a flying drone to follow a trail in the forest. 
Geometric and appearance methods can also be used together to train a traversability classifier. Delmerico et al.\cite{delmerico2017onthespot} extended the previous work \cite{delmerico2016active} by proposing the first on-the-spot training method that uses a flying drone to gather the data and train an estimator in less than 60 seconds. 

Data can also extract in simulations, where an agent interacts in an artificial environment. Usually, no-wheel legged robot able to traverse harder ground, can benefits from data gathering in simulations due to the high availability. For example, Klamt et al. \cite{tobias2017anytime} proposed a locomotion planning that is learned in a simulation environment. 

On other major distiction we can made is between different types of robots: wheel and no-wheel. We will focus on the later since we adopt a legged crocodile-like robot to extend the existing framework proposed by Chavez-Garcia et al. \cite{omar2018traversability}.

Legged robots show their full potential in rough and unstructured terrain, where they can use a superior move set compared to wheel robots. Different frameworks have been proposed to compute safe and efficient paths for legged robots. Wermelinger et al. \cite{wermelinger2016navigation} uses typical map characteristics such as slopes and roughness gather using onboard sensors to train a planner. The planner uses a RRT* algorithm to compute the correct path for the robot on the fly. Moreover, the algorithm is able to first find an easy local solution and then update its path to take into account more difficult scenarios as new environment data is collected. 

Due to uneven shape rough terrain, legged robots must be able to correctly sense the ground to properly find a correct path to the goal. Wagner et al. \cite{wagner2016foot} developed a method to estimate the contact surface normal for each foot of a legged robot relying solely on measurements from the joint torques and from a force sensor located at the foot. This sensor at the end of a leg optically determines its deformation to compute the force applied to the sensor. They combine those sensors measurement in an Extended Kalman Filter (EKF). They showed that the resulting method is capable of accurately estimating the foot contact force only using local sensing.

While the previous methods rely on handcrafted map's features extraction methods to estimate the cost of a given patch using a specific function, new frameworks that automatise the features extraction process has been proposed recently. Lorenz et al. \cite{wellhausen2019where} use local sensing to train a deep convolutional neural network to predict terrain's properties. They collect data from robot ground interaction to label each image in front of the robot in order to predict the future interactions with the terrain showing that the network is perfectly able to learn the correct features for different terrains.
Furthermore, they also perform weakly supervised semantic segmentation using the same approach to divide the input images into different ground classes, such as glass and sand, showing respectable results.
\end{document}