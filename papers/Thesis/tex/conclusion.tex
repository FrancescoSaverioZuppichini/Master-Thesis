\documentclass[../document.tex]{subfiles}
\begin{document}
\chapter{Conclusions and future work}
\label{chap : conclusions}
\section{Conclusions}
We developed a framework for traversability estimation based entirely on simulation data and applicable to any ground robot. Simulation data allowed cheap, fast, quality and massive data gathered since 1) no real robot was used 2) simulations can be run faster than real-time and parallelize 3) maps can be engineered to maximize robots exploration and interactions with specific features 4) simulations can be run for any amount of time. 

We trained a classifier to learn the robot's locomotion by supervised learning directly on ground patches. We tested the pipeline on a legged crocodile-like robot. 
We used a deep convolutional neural network to regress and classify the patches. We have showed both methods to be effective and we selected the latter due to its better accuracy once a threshold is fixed. We quantitatively measured the model's performance using different numeric metrics. Then, we visualized the traversability probability on real-world terrain to qualitatively evaluate the network. 

Later, we interpreted the network's predictions applying several approaches. First, we proved that features were correctly aggregate and separate based on their classes. Then, we explored different samples from the test set by visualizing with part of the surfaces caused the model's prediction. We discovered that the model always analyzes the correct ground features. Finally, we analyzed the network's strength and robustness by comparing different custom patches with the real advancement computed in the simulator. The results showed that the model expressed a correct degree of uncertainty in some edge cases while been able to properly classify most samples.
\section{Limitations}
We mention three important limitations of our methodology. First, we did not take into account grounds material but only relied on the terrain's 3D as the only factor to predict traversability, with the ground activing as a rigid object. This is not true when the surface is composed of slippery or sticky elements (e.g. oil or mud) but is a realistic approximation in other scenarios composed by solid ground. The second limitation is given by the data gathering process. The performance of the model directly depends on the quality of the data gathered during simulation that may not be always adequate to replicate the real world scenarios. Lastly, not all of the robot's information is currently used to train the estimator. In the current state, we not include the legs position and we rely only on the torso's pose.  
\section{Future work}
The first two problems can be tackled by additional input to the model while training (e.g. a coefficient associated with the material of the terrain).  While to overcome the third, we could incorporate the pose of each leg by 1) encoding all the legs pose and including them as an input to the network  2) by  generating one patch for each limb and stacking them together.
\end{document}