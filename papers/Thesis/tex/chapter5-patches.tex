\documentclass[../document.tex]{subfiles}
\begin{document}
\section{Robustness}
\label{sec: robustness}
To conclude the investigation of the network's abilities, we tested the model's robustness on patches with different features, walls, bumps, ramps, and examined the model's predictions against the real robot advancement obtained from the simulator. In general, the model's outputs matched the ground truth. When the network failed, on some edge cases, it showed a certain degree of uncertainty.
\subsection{Untraversable walls at increasing distance from the robot}
We first test we performed was to place a not traversable wall in front of \emph{Krock} at gradually moving towards the end. At a certain distance, we expected the model's prediction to be traversable even if the wall itself is too tall. Why? Because the robot will be able to travel more than the threshold before being stopped by the obstacle.

We created fifty-five different patches by first placing a wall $16$cm width exactly in front of the robot and then move it by $1$cm at the time towards the end. Figure \ref{fig : walls-distance} shows some of the inputs patches ordered by distance from the robot. 
\begin{figure}[htbp]
    \centering
    \begin{subfigure}[b]{0.24\linewidth}
    \includegraphics[width=\linewidth]{../img/5/custom_patches/walls_front/all/00-3d.png}
    \end{subfigure}
    \begin{subfigure}[b]{0.24\linewidth}
    \includegraphics[width=\linewidth]{../img/5/custom_patches/walls_front/all/04-3d.png}
    \end{subfigure}
    \begin{subfigure}[b]{0.24\linewidth}
    \includegraphics[width=\linewidth]{../img/5/custom_patches/walls_front/all/08-3d.png}
    \end{subfigure}
    \begin{subfigure}[b]{0.24\linewidth}
    \includegraphics[width=\linewidth]{../img/5/custom_patches/walls_front/all/12-3d.png}
    \end{subfigure}
    \begin{subfigure}[b]{0.24\linewidth}
    \includegraphics[width=\linewidth]{../img/5/custom_patches/walls_front/all/16-3d.png}
    \end{subfigure}
    \begin{subfigure}[b]{0.24\linewidth}
    \includegraphics[width=\linewidth]{../img/5/custom_patches/walls_front/all/18-3d.png}
    \end{subfigure}
    \begin{subfigure}[b]{0.24\linewidth}
    \includegraphics[width=\linewidth]{../img/5/custom_patches/walls_front/all/21-3d.png}
    \end{subfigure}
    \begin{subfigure}[b]{0.24\linewidth}
    \includegraphics[width=\linewidth]{../img/5/custom_patches/walls_front/all/24-3d.png}
    \end{subfigure}
    \caption{Some of the tested patches with a non traversable walls at increasing distance from the robo's headt.}
    \label{fig : walls-distance}
    \end{figure}
The model's predictions are displayed in figure \ref{fig : wall-distance-preds}. We can see how the two classes invert their values around $20$cm. Moreover, the predictions, are consistent and do not change multiple times. Intuitively, if a wall is traversable from a certain distance, it will still be if we place even further.
\begin{figure}[htbp]
    \centering
\begin{subfigure}[b]{1\linewidth}
    \includegraphics[width=\linewidth]{../img/5/custom_patches/walls_front/predictions.png}
    \end{subfigure}
    \caption{Traversability probabilities for patches  with a non traversable walls at increasing distance from the robot.}
\label{fig  : wall-distance-preds}
\end{figure}
% Summarized by the following table:
% \begin{table} [htbp]
%     \centering
%     \begin{tabular}{l|cc}
%         Distance(cm) & Prediction \\ 
%         \hline
%         0 - 20  & Not traversable \\ 
%         22 - end & Traversable \\ 
%         \hline
%     \end{tabular}
%     \caption{Model predictions for patches  with a non traversable walls at increasing distance from the robot.}
% \end{table}
To be sure the results are correct, we run the last not traversable and the first traversable patch on the simulator to get real advancement. In the simulator, Krock advances $18.3$cm on the first, not traversable patch \ref{fig : walls-traversable-a} where the wall is at $20$cm from the robot's head. While, on the first traversable patch, figure ref{fig :walls-traversable-b}, with a wall at $22$cm, the robot was able to travel for $20.2$cm. Correctly, the network's predictions are supported by the ground truth obtained from the simulation. Even more important, the model understood that the distance from the obstacle is more relevant than its height.
\begin{figure}[htbp]
    \centering
    \begin{subfigure}[b]{0.33\linewidth}
        \includegraphics[width=\linewidth]{../img/5/custom_patches/walls_front/2-2d.png}
        \end{subfigure}   
    \begin{subfigure}[b]{0.33\linewidth}
        \includegraphics[width=\linewidth]{../img/5/custom_patches/walls_front/1-2d.png}
    \end{subfigure}   \\
    \begin{subfigure}[b]{0.33\linewidth}
        \includegraphics[width=\linewidth]{../img/5/custom_patches/walls_front/2-3d.png}
        \caption{Distance $20$cm}
    \label{fig : walls-traversable-a}
    \end{subfigure}   
    \begin{subfigure}[b]{0.33\linewidth}
        \includegraphics[width=\linewidth]{../img/5/custom_patches/walls_front/1-3d.png}
        \caption{Distance $22$cm}
    \label{fig : walls-traversable-b}
    \end{subfigure}   
    \caption{We run the last and first not traversable and traverable patch labeled by the model. Correctly, the real advancement is greater than the threshold when the distance between the robot is also greater.}
    \label{fig : walls-traversable}
\end{figure}
Furthermore, we increased the wall size of the first traversable patch, figure \ref{fig : walls-traversable-b}, to $10$ and to $50$m to see if the model will be confused. Accurately, the robot was not confused by the enourmous wall and the predictions did no change and those patches were still labeled as traversable. Figure \ref{fig : walls-tall} visualize those patches.
\begin{figure}[htbp]
    \centering
    \begin{subfigure}[b]{0.33\linewidth}
        \includegraphics[width=\linewidth]{../img/5/custom_patches/walls_front/big-1-2d.png}
    \caption{height $=10$m}
    \end{subfigure}   
    \begin{subfigure}[b]{0.33\linewidth}
        \includegraphics[width=\linewidth]{../img/5/custom_patches/walls_front/big-2-2d.png}
        \caption{height $=50$m}
    \end{subfigure}   
\caption{Two patches with a very tall wall at a distance major than the threshold. The model labels them as traversable without been confused by the huge obstacle.}    
\label{fig : walls-tall}
\end{figure}
\subsection{Walls at increasing height in front of the robot}
After we tested the distance from Krock's and a wall, we decided to fix the obstacle position but increase its heights.  We run forty patches in the simulator with a wall place exactly in front of the robot with a height from $1$cm to $20$cm. Figure \ref{fig : walls-height} shows some of the inputs.

\begin{figure}[htbp]
    \centering
    \begin{subfigure}[b]{0.24\linewidth}
    \includegraphics[width=\linewidth]{../img/5/custom_patches/walls_increasing/all/00-3d.png}
    \end{subfigure}
    \begin{subfigure}[b]{0.24\linewidth}
    \includegraphics[width=\linewidth]{../img/5/custom_patches/walls_increasing/all/03-3d.png}
    \end{subfigure}
    \begin{subfigure}[b]{0.24\linewidth}
    \includegraphics[width=\linewidth]{../img/5/custom_patches/walls_increasing/all/06-3d.png}
    \end{subfigure}
    \begin{subfigure}[b]{0.24\linewidth}
    \includegraphics[width=\linewidth]{../img/5/custom_patches/walls_increasing/all/09-3d.png}
    \end{subfigure}
    \begin{subfigure}[b]{0.24\linewidth}
    \includegraphics[width=\linewidth]{../img/5/custom_patches/walls_increasing/all/11-3d.png}
    \end{subfigure}
    \begin{subfigure}[b]{0.24\linewidth}
    \includegraphics[width=\linewidth]{../img/5/custom_patches/walls_increasing/all/14-3d.png}
    \end{subfigure}
    \begin{subfigure}[b]{0.24\linewidth}
    \includegraphics[width=\linewidth]{../img/5/custom_patches/walls_increasing/all/17-3d.png}
    \end{subfigure}
    \begin{subfigure}[b]{0.24\linewidth}
    \includegraphics[width=\linewidth]{../img/5/custom_patches/walls_increasing/all/19-3d.png}
    \end{subfigure}
    \caption{Some of the tested patches with a walls at increasing height ahead of Krock.}
\label{fig : walls-height}
    \end{figure}
The models predicted that the walls under $10$cm are traversable. We plotted the classes probabilities in figure \ref{fig : walls-height-preds} in which we cab observer the model's prediction change smoothly revealing a degree of uncertainty near the edge cases, $\approx 10$cm.
\begin{figure}[htbp]
    \centering
\begin{subfigure}[b]{1\linewidth}
    \includegraphics[width=\linewidth]{../img/5/custom_patches/walls_increasing/predictions.png}
    \end{subfigure}
    \caption{Traversability probabilities against walls height in front of Krock.}
\label{fig : walls-height-preds}
\end{figure}
We compared the model's prediction with the advancement computed in the simulator using the same approach from the last section. Figure \ref{fig : wall-height-sim} shows the results from the last traversable patch and the first non traversable.

\begin{figure}[htbp]
    \centering
    \begin{subfigure}[b]{0.33\linewidth}
        \includegraphics[width=\linewidth]{../img/5/custom_patches/walls_increasing/1-2d.png}
    \end{subfigure}   
    \begin{subfigure}[b]{0.33\linewidth}
        \includegraphics[width=\linewidth]{../img/5/custom_patches/walls_increasing/2-2d}
    \end{subfigure}   
    \begin{subfigure}[b]{0.33\linewidth}
        \includegraphics[width=\linewidth]{../img/5/custom_patches/walls_increasing/1-3d.png}
    \caption{height $9$cm}
    \end{subfigure}   
    \begin{subfigure}[b]{0.33\linewidth}
        \includegraphics[width=\linewidth]{../img/5/custom_patches/walls_increasing/2-3d}
        \caption{Height $10$cm}
    \end{subfigure}   
    \caption{We run the last and first traversable and not traverable patch labeled by the model. Correctly the model's prediction matches the advancement from the simulator.} 
    \label{fig : wall-height-sim}
\end{figure}
In the first case, the simulator outputted and advancement of $21.2$cm meaning that Krock was able to overcome the obstacle, while it failed in the second case. Correctly, the predictions matched the real data.

\subsection{Walls with increasing height and distance from the robot}
We combined the previous experiments and tested the model predictions against the ground truth for each height/distance combination. To reduce the number of samples and improve readability, we limited to consider only patches with a wall tall between $5$cm and $20$cm, we know from previous sections patches with a value smaller and bigger obstacle are traversable and not traversable respectively. Similar, we set the wall's distance from Krock's head between $1$cm to $30$cm for the same reasons. To evaluate the model's prediction, we run all the patches several times on the simulator and average the results. Figure \ref{fig :  walls-heights-preds} shows the models outputs. We highlighted the false positive and the false negative by red and blue respectively.  Since we spawned the robot directly on the patch inside the simulator, the outputs may change across different runs. Sometimes, two runs on the same patch can produce slightly different advancement due to some really small changes in the initial position of the robot caused the spawning procedure or some lag. Figure \ref{fig : walls-heights-advs} displays the real advancements average across six different runs for all the patches. Additionally, for completeness, we also display in figure \ref{fig : walls-heights-std} the variance between all simulation's runs to highlighted cases where the advancement changes the most across different runs:  

\begin{figure}[htbp]
    \centering
    \begin{subfigure}[b]{0.66\linewidth}
        \includegraphics[width=\linewidth]{../img/5/custom_patches/walls_heights/walls_heights_preds.png}
        \caption{Predictions}
        \label{fig : walls-heights-preds}
    \end{subfigure}   
    \begin{subfigure}[b]{0.66\linewidth}
        \includegraphics[width=\linewidth]{../img/5/custom_patches/walls_heights/walls_heights_advs.png}
        \caption{Averaged real advancement}
        \label{fig : walls-heights-advs}
    \end{subfigure}   
\end{figure}

    \begin{figure}[htbp]
        \centering
    \ContinuedFloat
    \begin{subfigure}[b]{0.66\linewidth}
        \includegraphics[width=\linewidth]{../img/5/custom_patches/walls_heights/walls_heights_std.png}
        \caption{Standard deviation of the real advancement across six different runs.}
        \label{fig : walls-heights-std}
    \end{subfigure}   
\caption{Results for all the combination of patches with a wall's distance from $2-30$cm and heights between $5 - 19$cm. False negatives are labeled with red while false positive with blue. The model failed to classify some edges cases when the wall is very close to Krock's head and when the wall's distance is near the threshold. The overall accuracy is $91\%$.}    
\label{fig : walls-heights}
\end{figure}
We obtained an overall accuracy of $91\%$. The model failed to predict some of the edge cases. The false positives are located in two regions: on the bottom left and on the top center. The first ones are the patches with a wall just ahead Krock of heights between $\approx 8 - 11$. The second region appears when the wall is at $\approx 20$cm, the threshold. But, even if the model failed to classify those inputs, it shows a correct degree of uncertainty. For example, most of the predictions' probability for the patches at $10$cm (red cells) are less than $0.7$, some of them even close to $0.5$.

The false negative, (blue), are the inputs at distance close to the threshold and with a wall height between traversable and not traversable. Even if the model wrongly classified some of the inputs, all those errors are in the edge cases where the predicted classes' probability is not maximum. Moreover, in most cases, the model's showed uncertainty, especially on the false negative. Also, the prediction changes smoothly without any spikes accordingly to the features of the terrain. This shows a correct degree of understanding of the surface inputs. For instance, If the model outputs not traversable at height of $10$cm and at a distance of $16$cm, then all the taller wall are correctly labeled as not traversable showing consistency and predictability. 
\subsection{Corridors}
We also tested the model against corridors. We expected the model fails due to the lack of train samples similar to corridors. Only two maps in the whole dataset have featers that could generate corridors. This is confirmed by presence of only few patches with corridors in the features space, figure \ref{fig : pca-patches-200}. We generated ten different terrains starting from two walls at $30$cm from the borders. Figure \ref{fig : tunnels}  visualizes the inputs.
\begin{figure}[htbp]
    \centering
    \begin{subfigure}[b]{0.24\linewidth}
    \includegraphics[width=\linewidth]{../img/5/custom_patches/tunnel/all/00-3d.png}
    \end{subfigure}
    \begin{subfigure}[b]{0.24\linewidth}
    \includegraphics[width=\linewidth]{../img/5/custom_patches/tunnel/all/01-3d.png}
    \end{subfigure}
    \begin{subfigure}[b]{0.24\linewidth}
    \includegraphics[width=\linewidth]{../img/5/custom_patches/tunnel/all/02-3d.png}
    \end{subfigure}
    \begin{subfigure}[b]{0.24\linewidth}
    \includegraphics[width=\linewidth]{../img/5/custom_patches/tunnel/all/03-3d.png}
    \end{subfigure}
    \begin{subfigure}[b]{0.24\linewidth}
    \includegraphics[width=\linewidth]{../img/5/custom_patches/tunnel/all/04-3d.png}
    \end{subfigure}
    \begin{subfigure}[b]{0.24\linewidth}
    \includegraphics[width=\linewidth]{../img/5/custom_patches/tunnel/all/06-3d.png}
    \end{subfigure}
    \begin{subfigure}[b]{0.24\linewidth}
    \includegraphics[width=\linewidth]{../img/5/custom_patches/tunnel/all/08-3d.png}
    \end{subfigure}
    \begin{subfigure}[b]{0.24\linewidth}
    \includegraphics[width=\linewidth]{../img/5/custom_patches/tunnel/all/09-3d.png}
    \end{subfigure}
    \caption{Some of the tested patches with corridors created by two walls at different distances.}
    \label{fig : tunnels}
\end{figure}
\begin{figure}[htbp]
    \centering
\begin{subfigure}[b]{1\linewidth}
    \includegraphics[width=\linewidth]{../img/5/custom_patches/tunnel/predictions.png}
    \end{subfigure}
    \caption{Traversability probabilities against corridor offset from the top borders.}
    \label{fig : tunnels-preds}
\end{figure}
The model scored an accuracy of just $40\%$. It wronly predicted that all patches expected the last two are traversable. In reality, only the first two patches are traversable, while all the others are not. When the distance between the walls reach a certain point, the robot was stuck and it was not able to move. Interesting, all the patches with corridor's width less than the robot's footprin represented an impossible situation, since the robot cannot go inside an obstacle. But, even some of those inputs were classified as traversable by the model. We computed the Grad-CAM output on last fours samples, figure \ref{fig : tunnels-grad-cam} visualizes it. In the first two, wrongly predicted as traversable, the front region was highlighted meaning that the network did not consider the distance between the wallsh. in the last two patches, the network discriminated booth the under and ahead inner part of the corridor. Clearly, the model was confused. To improve the network performance on corridors we incorporate new maps in the train set.
 \begin{figure}[htbp]
    \centering
    \begin{subfigure}[b]{0.24\linewidth}
    \includegraphics[width=\linewidth]{../img/5/custom_patches/tunnel/all/06-3d-grad.png}
    \end{subfigure}
    \begin{subfigure}[b]{0.24\linewidth}
    \includegraphics[width=\linewidth]{../img/5/custom_patches/tunnel/all/07-3d-grad.png}
    \end{subfigure}
    \begin{subfigure}[b]{0.24\linewidth}
    \includegraphics[width=\linewidth]{../img/5/custom_patches/tunnel/all/08-3d-grad.png}
    \end{subfigure}
    \begin{subfigure}[b]{0.24\linewidth}
    \includegraphics[width=\linewidth]{../img/5/custom_patches/tunnel/all/09-3d-grad.png}
    \end{subfigure}
    \caption{Fours patches with corridor's width smaller than Krock's footprint making impossible for the robot to traverse them. Only the last to surfaces were correct label as not traversable.}
    \label{fig : tunnels-grad-cam}
\end{figure}
\subsection{Ramps}
We generated twenty ramps with a maximum height from $0.25$m to $4$m. Figure \ref{fig: ramps} shows some of the inputs. The model labeled as traversable the surfaces with a maximum height less than $1$m and not traversable the others, figure \ref{fig : ramps-preds} plots the traversability probabilities against the maximum height of each ramp. Then, we tested the last traversable patch and the first not traversable with the real advancement gather from the simulator. Figure \ref{fig : ramps-sim} shows that the model's predictions are confirmed by the ground truth values. 
\begin{figure}[htbp]
    \centering
    \begin{subfigure}[b]{0.24\linewidth}
    \includegraphics[width=\linewidth]{../img/5/custom_patches/ramp/all/00-3d.png}
    \end{subfigure}
    \begin{subfigure}[b]{0.24\linewidth}
    \includegraphics[width=\linewidth]{../img/5/custom_patches/ramp/all/03-3d.png}
    \end{subfigure}
    \begin{subfigure}[b]{0.24\linewidth}
    \includegraphics[width=\linewidth]{../img/5/custom_patches/ramp/all/06-3d.png}
    \end{subfigure}
    \begin{subfigure}[b]{0.24\linewidth}
    \includegraphics[width=\linewidth]{../img/5/custom_patches/ramp/all/09-3d.png}
    \end{subfigure}
    \begin{subfigure}[b]{0.24\linewidth}
    \includegraphics[width=\linewidth]{../img/5/custom_patches/ramp/all/11-3d.png}
    \end{subfigure}
    \begin{subfigure}[b]{0.24\linewidth}
    \includegraphics[width=\linewidth]{../img/5/custom_patches/ramp/all/14-3d.png}
    \end{subfigure}
    \begin{subfigure}[b]{0.24\linewidth}
    \includegraphics[width=\linewidth]{../img/5/custom_patches/ramp/all/17-3d.png}
    \end{subfigure}
    \begin{subfigure}[b]{0.24\linewidth}
    \includegraphics[width=\linewidth]{../img/5/custom_patches/ramp/all/19-3d.png}
    \end{subfigure}
    \caption{Some of the tested patches with steep ramps.}
    \label{fig: ramps}
    \end{figure}
\begin{figure}[htbp]
    \centering
    \begin{subfigure}[b]{1\linewidth}
    \includegraphics[width=\linewidth]{../img/5/custom_patches/ramp/predictions.png}
    \end{subfigure}
    \caption{Traversability probabilities against maximum height of each ramp.}
    \label{fig: ramps-preds}
\end{figure}
\begin{figure}[htbp]
    \centering
    \begin{subfigure}[b]{0.33\linewidth}
        \includegraphics[width=\linewidth]{../img/5/custom_patches/ramp/ramp-6-2d.png}
    \end{subfigure}   
    \begin{subfigure}[b]{0.33\linewidth}
        \includegraphics[width=\linewidth]{../img/5/custom_patches/ramp/ramp-7-2d}
    \end{subfigure}   
    \begin{subfigure}[b]{0.33\linewidth}
        \includegraphics[width=\linewidth]{../img/5/custom_patches/ramp/ramp-6-3d.png}
    \caption{Height $0.94$cm}
    \end{subfigure}   
    \begin{subfigure}[b]{0.33\linewidth}
        \includegraphics[width=\linewidth]{../img/5/custom_patches/ramp/ramp-7-3d}
        \caption{height $1.1$m}
    \end{subfigure}   
\caption{The last traversable and the first non traversable patches with a steep ramp ahead of Krock.}    
\label{fig: ramps-sim}
\end{figure}
\end{document}