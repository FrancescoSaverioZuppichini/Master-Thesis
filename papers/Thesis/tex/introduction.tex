\documentclass[../document.tex]{subfiles}
\begin{document}
\chapter{Introduction}
Effective identification of traversable terrain is essential to operate mobile robots in every type of environment. Today, there two main different approaches used in robotics to properly navigate a robot: online and offline. The first one uses local sensors to map the surroundings "on the go" while the second equip the mobile robot with an already labeled map of the terrain. 

In most indoor scenarios, specific hardware such as infrared or lidar sensors is used to perform online mapping while the robot is exploring, this is the case of the most recent vacuum cleaner able to map all the rooms in an apartment. With the recent breakthroughs of deep learning in computer vision, more and more cameras have been used in robotics. For example, self-driving cars utilize different cameras around the vehicle to avoid obstacle using object detection.
Indoor scenarios share similar features across different places shifting the problem from which ground can be traversable to which obstacle must be avoided. For instance, the floor is always flat in almost all rooms due to is artificial design. Moreover, usually, traversability must be estimated on the fly due to the high number of possible obstacles and to the layout of the objects in each room may not be persistent in time. 

On the other hand, outdoor scenarios may have less artificial obstacle but their homogeneous ground makes challenging to determine where the robot can properly travel. Moreover, a given portion of the ground may not be traversable by all direction due to the not uneven terrain. Fortunately, a height map of the ground can be obtained easily by using third-party services or special flying drones.  Those maps are extremely valuable in robotics applications since they provide an efficient way to examine the features of terrains, such as bumps, holes, and walls.

These scenarios have different difficulties. In indoor environments, it is easier to move the robot on the ground since it is designed for humans, but harder to perform obstacle
avoidance. While in outdoors scenario it is maybe more challenged to the first estimate where the robot can go due to the huge variety of ground features that may influence traversability. 
To learn where to move, an artificial controller must be trained to predict the robot interaction with the environment. Such a process requires to collect some data to train the controller. However, this may not be a straight forward process. 
In indoors terrain,  most of the times, data is collected by driving the robot directly in the environment by a human or an artificial controller. While in the outdoors scenario data is assembled using the simulation for convenience.

Our approach aims to estimate traversability of a legged robot krocodile-like robot called Krock. We generate different uneven grounds in form of height map and let the robot walk for a certain amount of time on each one of them while recording its interaction, position and orientation. After, for each stored robot position we crop the corresponding ground portion in which the robot was during the simulator, those patches composed the training dataset. We select a minimum space in a fixed amount of time that the robot must travel to successfully traverse a ground region and use it to label the dataset. Then, we fit a deep convolutional neural network to predict the traversability probability. Later, we evaluate it using different metrics using real-world terrains.

The report is organized as follow, the next chapter introduces the related work, Chapter 2 describes our approach, Chapter 3 talks in deep about the implementation details,
Chapter 4 shows the results and Chapter 5 discuss conclusion and future work.
% \cite{einstein} asddsa
\end{document}