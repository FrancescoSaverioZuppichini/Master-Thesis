\documentclass[../document.tex]{subfiles}
\begin{document}
\section{Introduction}
Effective identification of traversable terrain is essential to operate mobile robots in every type of environment. Today, there two main different approaches used to a robot: online and offline. The first one uses local sensors to map the surroundings while the second equip the mobile robot with an already labeled map of the terrain. 

In most indoor scenarios, specific hardware such as infrared or LIDAR sensors is used to perform online mapping while the robot is exploring, this is the case of the most recent vacuum cleaner. In the last years, thank to the new application of deep learning in computer vision, more and more cameras have been used in robotics. For example, self-driving cars utilise different cameras around the vehicle to avoid obstacle using object detection. Indoor scenarios share similar features across different places shifting the problem from which ground can be traversable to which obstacle must be avoided. For example, the floor is always flat without any bumps or holes due to is artificial design. Moreover, usually traversability must be estimated on the fly due to the high number of possible obstacles and to the layout of the objects in each room may not be persistent in time. 

On the other hand, outdoor scenarios may have less artificial obstacle but a no homogeneous ground making challenging to estimate where the robot can properly travel. Moreover, a given portion of the ground may not be traversable by all direction due to the not uneven terrain. Fortunately, a height map of the ground can be obtained easily by using third-party services or special flying drones. 

These two different scenarios have different challenges, for example, in control environments such us indoor it is easier to move the robot on the ground but harder to perform obstacle
avoidance. While in outdoors scenario it is hard to first estimate where the robot is able to move due to possible ground features that may influence traversability. Furthermore, data gathering may not be straight forward. In indoors terrain, data is collected most of the times by driving the robot directly in environment by a human or an artificial controller. While in the outdoors scenario data is collected using simulation for convenience.

Our approach aims to estimate traversability on uneven ground, mostly outdoors scenarios. Our frameworks have two main phases, we first run different simulations by spawning the robot on synthetically
created maps in different terrains with walls, bumps, and slopes. We record the robot position and orientation, then we crop a region from the corresponding height map such that it includes the robot's footprint and a portion of ground This value is calculated by observing the advancement of the given robot on flat ground and taking its half. So, each patch contains the current robot position on the map and the future position if the robot will move at the maximum speed. Later, to create the dataset for the classifier, we label each patch using a threshold. Then, we train a deep convolutional neural network to fit the dataset. Later, we evaluate it using different metrics using real world terrains.

The report is organized as follow, the next chapter introduces the related work, Chapter 2 describes our approach, Chapter 3 talks in deep about the implementation details,
Chapter 4 shows the results and Chapter 5 discuss conclusion and future work.
% \cite{einstein} asddsa
\end{document}
