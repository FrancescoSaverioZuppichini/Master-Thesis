\documentclass[../document.tex]{subfiles}
\begin{document}
\chapter{Introduction}

All living beings, including humans, need to traverse ground to eat, sleep and breed. Anymals through millenia of evolution have developed different techniques to classify ground regions. Usually, local sensors are used to map new terrain to effectively navigate the environment. Humans, for instance, are equipped with a powerful traversability estimator, called vision, that is able to extract features from a local region, a patch, such as elevation, size, and steepness and combine them with the brain to produce a local planner able to find a suboptimal path to cross.

Similarly, ground robots need to walk in order to fulfill their tasks. Taking inspiration from nature, robots must be able to correctly aggregate terrain features to predict their traversability. Two main approaches have emerged over the years to collect the data, geometric and appearance. The first one utilize purely geometric features to label a patch, while the second relies on label categories, glass, rocks, etc, with fixed traversability probabilities extract mostly on camera's images. In both cases, those samples are used in supervised learning to train a traversability estimator. While appearance data must be collected directly from a real or simulated environment, geometric features do not directly depend on the robot's interactions with the ground. 

In most indoor scenarios, data is collected trough specific hardware, such as infrared sensors, cameras or LIDAR, during a robot exploration of the enviroment. Due to their artificial design, indoor enviroments shared similar features across the world, flatness of the floor, stairs and ceil. So, even if an agent is trained with samples collected in one location, due to physical limitations, the learned mapping can be effective in different locations. Moreover, in indoor enviroment, traversability estimation is mostly solved with obstacle avoidance since, by design, the floor alone do not include bumps, ramps or holes. In addition, while the robot operates there may be other agents interacting with the enviroment, such as humans, therefore a correct function to avoid collision must be properly learn to effective move safely the robot.

On the other hand, outdoor grounds may have less artificial obstacle but they have a not homogeneous ground making challenging to estimate where the robot can properly travel. Also, a given patch with a specific shape may not be traversable by all direction due to the not uniform features of the terrain. On top of that, using real world data may be unfeasible due to the time required to move the robot on different grounds and the possibily to introduce bias if the data samples are not varied enough. Fortunately, ground can be generated articially and used in a simulated enviroment to let the robot walk in it while storing its interactions. Moreover, grounds maps can be easily obtained nowadays by using third-party services such as google maps or flying drone equiped with mapping technologies.

We proposed a full pipeline to estimate traversability tested on legged crocodile-like robot on uneven terrain based entirely on data gather trought simulation. Simulation offers several advantages than real world data collection. First, we can include a rich array of different terrains without the need to physically move them. Data is collected faster and multiplice simulations can be launch in parallel keeping the cost of the real world hardware low. We first generated thirty synthetic maps encoded as heghtmaps with different features: bumps, walls, slopes, steps and holes. Then, we run the robot on each one of them for a certain amount of time while storing its interactions with the enviroment, position and orientation. Later, the procede to create the dataset by cropping a patch for each stored data point in a such way that includes the robot's footprint and the maximum possible amount of ground that can traverse in a selected time window. After, we label each patch by traversable or not traversable if the robot's advancement computed in that time using the same time window is less or greater than a treshold. The treshold depends on the robot's locomotion and is calculated by spawing it in front of different obstacle and observing its advancement, we set the value for the tested robot to twenty centiments. Those patches are used to fit a deep convolutional neural network to prediction the traversability leaving to the network the task to extract important ground's features.

The report is organized as follow, the next chapter  introduces the related work, Chapter 2 describes our approach, Chapter 3 talk in deep about the implementation details,
Chapter 4 shows the results and Chapter 5 discuss clusion and future work
\end{document}