\documentclass[../document.tex]{subfiles}
\begin{document}
\begin{abstract}
 Effective identification of traversable terrain is essential to operate mobile robots in different environments. 
    Historically, to estimate traversability, texture and geometric features were extracted before train a traversability estimator in a supervise way. However, with the recent deep learning breakthroughs in computer vision, terrain features may be learned directly from raw data, images or heightmaps, with an higher accuracy.
    
    We succesfully train a deep convolutional neural network to predict ground traversability for a legged crocodile-like robot with high accuracy given a ground patch. 
    The dataset was entiraly generated using simlation by first creating thirty synthetic synthetic maps with different grounds features such as slopes, holes, bumps, and walls. Then, we let the robot walk each map on for a certain amount of time on  while storing its intereactions. Later, precisely crop a patch from each simulation trajectory's point that includes the whole robot's footprint and the amount of ground it should reach without obstacles in a decided time window. Then, we label each patch as traversable or not traversable based on a minimum advancement of twenty centimenters. 
    
    We select and test different networks architecture and select the best performing one. We quantity shows its performance with different numeric metrics on different real-world terrains. Also, we qualitative evaluate the network by showing the predicted traversability probability on those grounds. 
    
    Later, we utilize model interpretability techniques to understand which patches confuse the network. We visualize the ground regions that the network fails to classify and visualize which part of the inputs was responsible for the wrong predictions. Finally, we test the model strength and robustness by comparing its prediction on custom patches composed by crafted features, such as a patch with a wall ahead, to the ground truth obtained by running again Krock on that ground in the simulator. 

\end{abstract}
\end{document}
% \cite{einstein} 
