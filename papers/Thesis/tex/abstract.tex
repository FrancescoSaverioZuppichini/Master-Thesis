\documentclass[../document.tex]{subfiles}
\begin{document}
\begin{abstract}
    Effective identification of traversable terrain is essential to operate mobile robots in different environments. 
    Historically, to estimate traversability, texture and geometric features were extracted before train a traversability estimator in a supervised way. However, with the recent deep learning breakthroughs in computer vision, terrain features can be learned directly from raw data, images or heightmaps, with higher accuracy.
    
    We implement a full pipeline to estimate traversability that first generates a dataset entirely through simulation and then it trains a deep convolutional neural network to predict traversability with high accuracy. The method is based on the framework proposed by Chavez-Garcia et al. \cite{omar2018traversability} and was originally tested on wheeled small robot. Collecting data in a simulation environment yields several advantages over the real world. The maps can be easily generated to include any features and maximize the robot exploration. Multiple simulations can be run in parallel reducing the cost for real-world hardware and increasing the dimension of the dataset. Real robots can be broken, affected by weather and failures.

    The dataset was generated using thirty synthetic surfaces with different features such as slopes, holes, bumps, and walls. Then, we let the robot walks on each map on for a certain amount of time while storing its interactions. Later, we precisely crop a patch from each simulation trajectory's point that includes the whole robot's footprint and the amount of ground it should reach without obstacles in a decided time window. Then, we label each patch as traversable or not traversable based on a minimum advancement depending on the robot used. The framework is open source and the same methodology can be adopted with any type of ground robots. We test the methodology on a legged crocodile-like robot that presents challenging locomotion.
    
    We proposed and test different residual networks architecture and select the best performing one. We quantity shows its performance with different numeric metrics on various real-world terrains. In addition, we qualitative evaluate the network by showing the predicted traversability probability on different grounds to better visualize the acquired knowledge.  
    
    Later, we utilize model interpretability techniques to understand the network's strengths and limitations. First, we show its ability to properly separate samples based on their features. Then we discover which patches confuse the network. We visualize the ground regions that the network fails to classify and find which portion of the inputs is responsible for the wrong predictions. Finally, we test the model strength and robustness by comparing its prediction on custom patches composed by crafted features, such as a patch with a wall ahead, to the ground truth obtained by running again Krock on that ground in the simulator. The results suggest that the model was able to match the ground truth in different situations.


\end{abstract}
\end{document}
% \cite{einstein} 
