\documentclass[../document.tex]{subfiles}
\begin{document}
\begin{abstract}
 Effective identification of traversable terrain is essential to operate mobile robots in the different environments.
    Historically, texture and geometric features were extracted by hand before train a traversability estimator in a supervise way. However, with the recent deep learning breakthroughs in computer vision, terrain features may be learned directly from raw data, images or heightmaps, with an higher accuracy.
    
    We trained a deep convolutional neural network on data gathered entirely through simulation to predict traversability for a legged crocodile-like robot called Krock. The training data was generated by letting the robot walking for a certain amount of time on thirty synthetic maps with different grounds features such as slopes, holes, bumps, and walls in a simulator environment while storing his pose, position, and orientation. We later used that information to precisely crop a patch from each simulation trajectory's point. This small portion of ground includes the whole robot's footprint and the ground it would reach using its maximum speed on flat ground. Each patch is label as traversable if Krock's advancement in that point, computed using a time window of two seconds, is greater than a threshold of twenty centimeters, not traversable otherwise. 
    
    We select and test different networks architecture and select the best performing one. We quantity shows its performance with different numeric metrics on different real-world terrains. Also, we qualitative evaluate the network by showing the predicted traversability probability on those grounds. 
    
    Later, we utilize model interpretability techniques to understand which patches confuse the most the network. We visualized the ground regions that the network fails to classify and visualize which part of the inputs was responsible for the wrong predictions. Finally, we test the model strength and robustness by comparing its prediction on custom patches composed by crafted features, such as a patch with a wall ahead, to the ground truth obtained by running again Krock on that ground in the simulator. 

\end{abstract}
\end{document}
% \cite{einstein} 
