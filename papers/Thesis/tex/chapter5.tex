\documentclass[../document.tex]{subfiles}
\begin{document}
\todo[inline]{This chapter is under development!}
\chapter{Interpretability}
\label{chap: interpretability}
Understanding the model's strength, robustness and, limitations is crucial to gain a better understanding of its outputs. This is even more important when working with controllers that can be deployed in real scenarios. In this section, we evaluated the quality of our traversability estimator with different methodology. First, we showed that the model has correctly learned grounds' features and was able to separable terrains based on them. Second, we utilized the challenging Quarry dataset to visualize the most traversable, the lest traversable and the misclassified patches.  Utilizing a special method, we determined that the model always looked at the correct features in the ground even if when it fails. Lastly, we crafted patches with different unique features, such as walls, bumps, etc, to test the robustness of the model by comparing its outputs to the real data gathered from the simulator. 


\subsection{Features separability}
In general, convolutional neural networks learn to encode images by applying filters of increasing size at each layer. Usually, the first layers learn basic features, such as edges, while the final one encodes complex shapes. The final convolutional layer's outputs are usually referred to as features space, consequently, a feature vector is just the output of the last layer for a given image. Those last features are combined and mapped to the correct classes by one or more fully connected. The following image help to visualize the different features learned at each layer. It was generated by plotting the learned features for different categories at different layers by 
Lee et al. \cite{deepbelief}. 
\begin{figure} [htbp]
    \centering
    \includegraphics[width=\linewidth]{../img/5/deep_belief.png}
    \caption{Figure from Lee et al. \cite{deepbelief} paper where they showed for different classes the low-level features (up) and the high-level features (down) learned by a convolution neural network.}
\end{figure}
So, a correctly trained network should be able to separate those features based on predicted classes. Intuitively, given two classes $\mathcal{A}$ and $\mathcal{B}$, for example, \emph{cat} and \emph{dog}, the high-level features for each class should not be the same, otherwise, the model may misclassify the input due to the overlap of different classes' features. 
One technique to discover the degree of separability of our network is to directly visualize the inputs features vectors for each class. Unfortunately, like most models, our network, MicroResNet, has a high dimensional feature space. Each patch is mapped to a  $[128, 3, 3 ]$ vector. We cannot directly visualize a $128$ dimension space, we reduced each feature vector dimension to a two-dimension by applying Principle Analysis Component (PCA) \cite{pca} to visualize it. We investigate the features space of the model booth in the train and test set.

\subsection{Train set}
The following figures shows the features of $11$K images sampled from the train set labeled with their classes, \emph{traversable} and \emph{not traversable}. 
\begin{figure} [htbp]
    \centering
    \begin{subfigure}[b]{1\textwidth}
        \includegraphics[width=\linewidth]{../img/5/pca/pca.png}
    \end{subfigure}
    \begin{subfigure}[b]{0.48\textwidth}
        \includegraphics[width=\linewidth]{../img/5/pca/pca-0.png}
        \caption{Not Traversable}
    \end{subfigure}
    \begin{subfigure}[b]{0.48\textwidth}
        \includegraphics[width=\linewidth]{../img/5/pca/pca-1.png}
        \caption{Traversable}
    \end{subfigure}
\caption{Principal Component Analysis on the features space computed using the features from the last convolutional layers on the training dataset.}
\end{figure}
We can clearly recognize two main clusters based on the labels' color, one on the left and one of the right. Those points are easily separable, even by human eyes, meaning that the model was able to learn meaning features from the dataset.  To be sure the center of each class' point cloud is not overlapping we plotted the density of each cluster.
\begin{figure} [htbp]
    \begin{subfigure}[b]{0.48\textwidth}
        \includegraphics[width=\linewidth]{../img/5/pca/pca-0-density.png}
        \caption{Not Traversable}
    \end{subfigure}
    \begin{subfigure}[b]{0.48\textwidth}
        \includegraphics[width=\linewidth]{../img/5/pca/pca-1-density.png}
        \caption{Traversable}
    \end{subfigure}
    \caption{Density plot for the points sampled from the training dataset in the features space. The centers of the cluster are not overlapping yielding a good separability and correct learning.}
    \end{figure}
Clearly, there is some distance between the centers. Furthermore, we can directly plot the patch corresponding to each feature vector to identify clusters of patches based on their position. Intuitively, if similar inputs are close to each other in the features space then the model also learned to effectively encode terrains features.   We decided to not show all images on the same plot to avoid overcrowding the image. Instead, we clustered the points using K-Means with $k=200$ clusters and then we took the patch that corresponded to the center point in each cluster. In this way, even if we are showing only a few inputs, we included all the meaningful features. The following image shows the result. 
\begin{figure} [htbp]
    \centering
    \begin{subfigure}[b]{1\textwidth}
        \includegraphics[width=\linewidth]{../img/5/pca/pca-patches-200.png}
    \end{subfigure}
    \caption{Patches plotted using the coordinate of the features vector obtained from the last convolutional layer's output and then reduced using PCA to a two-dimensional vector. Similar grounds are close to each other.}
\end{figure}
Definitely, patches with similar features are close to each other yielding a quality features encoding. On the left-top side, we can distinguish highly untraversable patches with walls/bumps in front of the robot. Going down, we encounter patches with smaller obstacles. On the plateau, there are traversable patches with small obstacles such as light bumps. Importantly,  those patches are the closest ones to the not traversable ones, so they were the hardest to separate, thus, to classify.  Going up on the right side, we see some grounds with small steps. Finally, on the top, we find all the downhill patches, the simplest ones to traverse.

\subsection{Test set}
We can apply the same procedure on the test set. Since it is a real world quarry, this dataset is harder than the train set and present challenging situations for the robot. The following image shows the features space after reducing its dimension to two using PCA.
\begin{figure} [htbp]
    \centering
    \begin{subfigure}[b]{1\textwidth}
        \includegraphics[width=\linewidth]{../img/5/pca/pca-test.png}
    \end{subfigure}
    \begin{subfigure}[b]{0.48\textwidth}
        \includegraphics[width=\linewidth]{../img/5/pca/pca-test-0.png}
    \end{subfigure}
    \begin{subfigure}[b]{0.48\textwidth}
        \includegraphics[width=\linewidth]{../img/5/pca/pca-test-1.png}
    \end{subfigure}
\caption{TODO}
\end{figure}
Interesting, the traversable patches are very near to each other, while the others span a very big surface. This suggests that there are many not traversable terrains with different features. The traversable points are clustered near the center, this implies that most of them share similar features. We plotted the density for each class to better understand where the most points are mapped.
\begin{figure} [htbp]
\begin{subfigure}[b]{0.48\textwidth}
    \includegraphics[width=\linewidth]{../img/5/pca/pca-test-0-density.png}
    \caption{Not Traversable}
\end{subfigure}
\begin{subfigure}[b]{0.48\textwidth}
    \includegraphics[width=\linewidth]{../img/5/pca/pca-test-1-density.png}
    \caption{Traversable}
    \label{fig : pca-test-density-1}
\end{subfigure}
\caption{Density plot for the points sampled from the test dataset in the features space. The centers of the cluster are not overlapping yielding a good separability and correct learning.}
\end{figure}
The two centers are really close to each other, making those samples harder to separate and some not traversable points are mixed up with the traversable ones. This explains the elevated number of false negative that lower down the AUC score on this dataset. We can also visualize the patches by plotting them using their features coordinates
\begin{figure} [htbp]
    \centering
    \begin{subfigure}[b]{1\textwidth}
        \includegraphics[width=\linewidth]{../img/5/pca/pca-test-patches-200-None-test.png}
    \end{subfigure}
\caption{Patches that correspond to coordinates in the features space of the last convolutional layers on the test dataset. Similar grounds are close to each other.}
\ref{fig: pca-test-patches}
\end{figure}
On the top left, from the not traversable cloud, we can see patches with a high level of bumps. Going down we find surfaces with huge walls in front of the robot while going close to the center we start to see all the traversable patches. Those samples have not too steep slopes. If we move to the density center, green double shown in figure \ref{pca-test-density-1}, we encounter lots of flat patches with little obstacles. Going up on the right branch we find downhill and on the top there are falls. 

In the following section, we will take a deep look at the test set to find which patches confuse the model. Most probably, those samples will be located between the two clusters center where the difference between classes' features is minimum.
\section{Grad-CAM}
Gradient-weighted Class Activation Mapping (Grad-CAM) \cite{gradcam} is a technique to produce \"visual explanations\" for convolutional neural networks. It highlights the regions of the input image that contribute the most to the predictions. 
\begin{figure} [htbp]
    \centering
    \begin{subfigure}[b]{1\textwidth}
        \includegraphics[width=\linewidth]{../img/5/grad_cam1.png}
    \end{subfigure}
\caption{Grad-CAM procedure on an input image. Image from the original paper \cite{gradcam}.}
\end{figure}
In detail, the output with respect to a target class is backpropagated while storing the gradient and the output in the last convolution. Then, a global average is applied to the saved gradient keeping the channel dimension in order to get a 1-d tensor, this will represent the importance of each channel in the target convolutional layer. After, each element of the convolutional layer outputs is multiplied with the averaged gradients to create the grad cam. This whole procedure is fast and it is architecture independent.

% \documentclass[../document.tex]{subfiles}
% \begin{document}
\section{Test dataset exploration}
\label{sec: quarry-dataset}
After showing the model's capability of correctly separate classes' features we utilized Grad-CAM to visualize some patches inside the quarry dataset. The aim of this section is to show how the model looks at meaningful features in each input to make the prediction even when it fails. For instance, imagine we feed to the model a not traversable patch with an obstacle and the network label is as traversable. Clearly, the output is wrong but the model's error may be caused by two different situations. The first one, the model could just have ignored the obstacle and looked away, meaning that it was not even able to understand there was an obstacle in the first place. Second, the network could have correctly look at the obstacle but thought that obstacle is traversable, showing the correct ability to find and use important features in the map. In the following sections, we showed that, even when the predictions are wrong, our model always look at the most important features of each input to determine its traversability. 

We divided the dataset patches into four classes based on the model's performance: worst, best, false positive and false negative. Then, we took twenty inputs from these sets and applied a technique called Grad-CAM to visualize which part of the ground the model is looking. These regions are highlighted in 3D render to better visualize which spot of the inputs caused the prediction. Before starting the exploration, let us introduce the method that allowed us to identify regions of interest on the patches. 
% \subsection{Best}
% We start evaluating our model by using the test set composed by samples from the Quarry map. Grad-CAM is an extremely powerful tool to interpret the results in this situation since most of the patches are hard to classify even at humans eye since they include different features on the same surface.
\subsection{Grad-CAM}
Before starting the exploration, let us introduce the method used to identify region of interest on the patches. Gradient-weighted Class Activation Mapping (Grad-CAM) \cite{gradcam} is a technique to produce \"visual explanations\" for convolutional neural networks. It highlights the regions of the input image that contribute the most to the predictions. For example, given a classifier able to recognize cats and cat image, Grad-CAM will point outs the cat.
\begin{figure} [htbp]
    \centering
    \begin{subfigure}[b]{1\textwidth}
        \includegraphics[width=\linewidth]{../img/5/grad_cam1.png}
    \end{subfigure}
\caption{Grad-CAM procedure on an input image. Image from the original paper \cite{gradcam}.}
\label{fig: gradcam}
\end{figure}
In detail, the output with respect to a target class is backpropagated while storing the gradient and the last convolution layer's output. Then, a global average is applied to the saved gradient keeping the channel dimension in order to get a 1-d tensor, this will represent the importance of each channel in the target convolutional layer. After, each element of the convolutional layer outputs is multiplied with the averaged gradients to create the grad cam. This whole procedure is fast and it is architecture independent. The procedure is summarized by figure \ref{fig : gradcam}

\subsection{Traversable patches}
These samples are the patches correctly predicted as traversable. We plotted twenty different inputs sampled uniformly according to the robot advancement to include as many interesting situations as possible. We recognized two main clusters of images: flat, figures \ref{fig : quarry-best-1},  \ref{fig : quarry-best-2},  \ref{fig : quarry-best-6},  \ref{fig : quarry-best-7},  \ref{fig : quarry-best-8},  \ref{fig : quarry-best-9},  \ref{fig : quarry-best-10},  \ref{fig : quarry-best-11},  \ref{fig : quarry-best-15},  \ref{fig : quarry-best-17},  \ref{fig : quarry-best-19}, and slopes, figures  \ref{fig : quarry-best-3},  \ref{fig : quarry-best-4},  \ref{fig : quarry-best-5},  \ref{fig : quarry-best-12},  \ref{fig : quarry-best-13},  \ref{fig : quarry-best-14},  \ref{fig : quarry-best-16},  \ref{fig : quarry-best-20}. The models is mostly interested in the left part of the patches with uneven ground on the left region, \ref{fig : quarry-best-1}, \ref{fig : quarry-best-2}, \ref{fig : quarry-best-3}, \ref{fig : quarry-best-4}, \ref{fig : quarry-best-5}, \ref{fig : quarry-best-7}, \ref{fig : quarry-best-10}, \ref{fig : quarry-best-12}, \ref{fig : quarry-best-13}, \ref{fig : quarry-best-14}, \ref{fig : quarry-best-15}, \ref{fig : quarry-best-16}, \ref{fig : quarry-best-18}. This is due to the fact that an obstacle in that region can block the rear legs and prevent the robot to advance. In other patches where most probably a untraversable features may also be present ahead, the model discriminates different parts of the map. For instance, figure \ref{fig : quarry-best-4} shows an interest region on the slope near the end and figure  \ref{fig : quarry-best-12} on two stop of the first faily big step ahead of the robot. Similarly, also figures  \ref{fig : quarry-best-14},  \ref{fig : quarry-best-17},  \ref{fig : quarry-best-19}.

In some situations, the model identify a possible untraversable spot only forward being sure the robot is no immeaditely stopped. There are two obvious cases, figures \ref{fig : quarry-best-8} ,  \ref{fig : quarry-best-9} and \ref{fig : quarry-best-19}. The first one is a totally flat surface, so the model looks as far as possible to check if there are obstacles. We have a similar situation in the last two patches, a surface with a some noise and a big bump respectively, where the network verifies these spots. So, rightly, the network analysis the first region of the patch that may contain an untraversable obstacle. 
\begin{figure}[H]
    \centering
    \begin{subfigure}[b]{0.192\linewidth}
    \includegraphics[width=\linewidth]{../img/5/quarry/best/20-patch-3d-majavi-colormap-0.png}
    \caption{0.20cm}
    \label{fig : quarry-best-1}
    \end{subfigure}
    \begin{subfigure}[b]{0.192\linewidth}
    \includegraphics[width=\linewidth]{../img/5/quarry/best/25-patch-3d-majavi-colormap-10.png}
    \caption{0.26cm}
    \label{fig : quarry-best-2}
    \end{subfigure}
    \begin{subfigure}[b]{0.192\linewidth}
    \includegraphics[width=\linewidth]{../img/5/quarry/best/30-patch-3d-majavi-colormap-20.png}
    \caption{0.30cm}
    \label{fig : quarry-best-3}
    \end{subfigure}
    \begin{subfigure}[b]{0.192\linewidth}
    \includegraphics[width=\linewidth]{../img/5/quarry/best/34-patch-3d-majavi-colormap-30.png}
    \caption{0.35cm}
    \label{fig : quarry-best-4}
    \end{subfigure}
    \begin{subfigure}[b]{0.192\linewidth}
    \includegraphics[width=\linewidth]{../img/5/quarry/best/38-patch-3d-majavi-colormap-40.png}
    \caption{0.38cm}
    \label{fig : quarry-best-5}
    \end{subfigure}
    \begin{subfigure}[b]{0.192\linewidth}
    \includegraphics[width=\linewidth]{../img/5/quarry/best/41-patch-3d-majavi-colormap-50.png}
    \caption{0.41cm}
    \label{fig : quarry-best-6}
    \end{subfigure}
    \begin{subfigure}[b]{0.192\linewidth}
    \includegraphics[width=\linewidth]{../img/5/quarry/best/44-patch-3d-majavi-colormap-60.png}
    \caption{0.44cm}
    \label{fig : quarry-best-7}
    \end{subfigure}
    \begin{subfigure}[b]{0.192\linewidth}
    \includegraphics[width=\linewidth]{../img/5/quarry/best/46-patch-3d-majavi-colormap-70.png}
    \caption{0.47cm}
    \label{fig : quarry-best-8}
    \end{subfigure}
    \begin{subfigure}[b]{0.192\linewidth}
    \includegraphics[width=\linewidth]{../img/5/quarry/best/49-patch-3d-majavi-colormap-80.png}
    \caption{0.49cm}
    \label{fig : quarry-best-9}
    \end{subfigure}
    \begin{subfigure}[b]{0.192\linewidth}
    \includegraphics[width=\linewidth]{../img/5/quarry/best/51-patch-3d-majavi-colormap-90.png}
    \caption{0.52cm}
    \label{fig : quarry-best-10}
    \end{subfigure}
    \begin{subfigure}[b]{0.192\linewidth}
    \includegraphics[width=\linewidth]{../img/5/quarry/best/54-patch-3d-majavi-colormap-100.png}
    \caption{0.54cm}
    \label{fig : quarry-best-11}
    \end{subfigure}
    \begin{subfigure}[b]{0.192\linewidth}
    \includegraphics[width=\linewidth]{../img/5/quarry/best/56-patch-3d-majavi-colormap-110.png}
    \caption{0.57cm}
    \label{fig : quarry-best-12}
    \end{subfigure}
    \begin{subfigure}[b]{0.192\linewidth}
    \includegraphics[width=\linewidth]{../img/5/quarry/best/58-patch-3d-majavi-colormap-120.png}
    \caption{0.59cm}
    \label{fig : quarry-best-13}
    \end{subfigure}
    \begin{subfigure}[b]{0.192\linewidth}
    \includegraphics[width=\linewidth]{../img/5/quarry/best/60-patch-3d-majavi-colormap-130.png}
    \caption{0.60cm}
    \label{fig : quarry-best-14}
    \end{subfigure}
    \begin{subfigure}[b]{0.192\linewidth}
    \includegraphics[width=\linewidth]{../img/5/quarry/best/62-patch-3d-majavi-colormap-140.png}
    \caption{0.62cm}
    \label{fig : quarry-best-15}
    \end{subfigure}
    \begin{subfigure}[b]{0.192\linewidth}
    \includegraphics[width=\linewidth]{../img/5/quarry/best/63-patch-3d-majavi-colormap-150.png}
    \caption{0.64cm}
    \label{fig : quarry-best-16}
    \end{subfigure}
    \begin{subfigure}[b]{0.192\linewidth}
    \includegraphics[width=\linewidth]{../img/5/quarry/best/65-patch-3d-majavi-colormap-160.png}
    \caption{0.66cm}
    \label{fig : quarry-best-17}
    \end{subfigure}
    \begin{subfigure}[b]{0.192\linewidth}
    \includegraphics[width=\linewidth]{../img/5/quarry/best/67-patch-3d-majavi-colormap-170.png}
    \caption{0.67cm}
    \label{fig : quarry-best-18}
    \end{subfigure}
    \begin{subfigure}[b]{0.192\linewidth}
    \includegraphics[width=\linewidth]{../img/5/quarry/best/68-patch-3d-majavi-colormap-180.png}
    \caption{0.68cm}
    \label{fig : quarry-best-19}
    \end{subfigure}
    \begin{subfigure}[b]{0.192\linewidth}
    \includegraphics[width=\linewidth]{../img/5/quarry/best/70-patch-3d-majavi-colormap-190.png}
    \caption{0.70cm}
    \label{fig : quarry-best-20}
    \end{subfigure}
    \label{fig : quarry-best}
    \caption{Traversable patches sampled from the test dataset correctly predicted by the model. We compute dthe Grad-CAM and applied as texture in the 3D render. The yellow highlights the ground region that contributed the most the model's prediction. }
    \end{figure}

\subsection{Non traversable patches}
    Not traversable patches correctly classified by the model. In this case, the region highlighted by the Grad-CAM more varied. Like in the previous section, in some patches, \ref{fig : quarry-worst-2}, \ref{fig : quarry-worst-6}, \ref{fig : quarry-worst-12}, \ref{fig : quarry-worst-14}, \ref{fig : quarry-worst-17}, \ref{fig : quarry-worst-19}, the non traversable features is directly under the robot. While, on other grounds with a atraversable left part but a big obstacle ahead, figures \ref{fig : quarry-worst-3}, \ref{fig : quarry-worst-5}, \ref{fig : quarry-worst-9}, \ref{fig : quarry-worst-15}, \ref{fig : quarry-worst-18}, the model identified the huge obstacle as the main reason for their untraversability. There mixed cases, figures \ref{fig : quarry-worst-2}, \ref{fig : quarry-worst-4}, \ref{fig : quarry-worst-10}, \ref{fig : quarry-worst-13}. Other grounds are mostly uneven, \ref{fig : quarry-worst-4}, \ref{fig : quarry-worst-11} and \ref{fig : quarry-worst-12} and the cam highlighted the biggest bumps. The last patch, \ref{fig : quarry-worst-8}, is very interesting. There is a trail with a obstacle parallel to it. Perfectly, the network identified one part obstacle as responsabile for the prediction. Probably, these spot is the points where the robot hitted the obstacle while going forward.

    \begin{figure}[H]
        \centering
        \begin{subfigure}[b]{0.192\linewidth}
        \includegraphics[width=\linewidth]{../img/5/quarry/worst/-35-patch-3d-majavi-colormap-0.png}
        \caption{-0.36cm}
        \label{fig : quarry-worst-1}
        \end{subfigure}
        \begin{subfigure}[b]{0.192\linewidth}
        \includegraphics[width=\linewidth]{../img/5/quarry/worst/-7-patch-3d-majavi-colormap-10.png}
        \caption{-0.08cm}
        \label{fig : quarry-worst-2}
        \end{subfigure}
        \begin{subfigure}[b]{0.192\linewidth}
        \includegraphics[width=\linewidth]{../img/5/quarry/worst/-4-patch-3d-majavi-colormap-20.png}
        \caption{-0.05cm}
        \label{fig : quarry-worst-3}
        \end{subfigure}
        \begin{subfigure}[b]{0.192\linewidth}
        \includegraphics[width=\linewidth]{../img/5/quarry/worst/-3-patch-3d-majavi-colormap-30.png}
        \caption{-0.03cm}
        \label{fig : quarry-worst-4}
        \end{subfigure}
        \begin{subfigure}[b]{0.192\linewidth}
        \includegraphics[width=\linewidth]{../img/5/quarry/worst/-2-patch-3d-majavi-colormap-40.png}
        \caption{-0.02cm}
        \label{fig : quarry-worst-5}
        \end{subfigure}
        \begin{subfigure}[b]{0.192\linewidth}
        \includegraphics[width=\linewidth]{../img/5/quarry/worst/-1-patch-3d-majavi-colormap-50.png}
        \caption{-0.01cm}
        \label{fig : quarry-worst-6}
        \end{subfigure}
        \begin{subfigure}[b]{0.192\linewidth}
        \includegraphics[width=\linewidth]{../img/5/quarry/worst/00-patch-3d-majavi-colormap-60.png}
        \caption{-0.01cm}
        \label{fig : quarry-worst-7}
        \end{subfigure}
        \begin{subfigure}[b]{0.192\linewidth}
        \includegraphics[width=\linewidth]{../img/5/quarry/worst/00-patch-3d-majavi-colormap-70.png}
        \caption{0.00cm}
        \label{fig : quarry-worst-8}
        \end{subfigure}
        \begin{subfigure}[b]{0.192\linewidth}
        \includegraphics[width=\linewidth]{../img/5/quarry/worst/00-patch-3d-majavi-colormap-80.png}
        \caption{0.01cm}
        \label{fig : quarry-worst-9}
        \end{subfigure}
        \begin{subfigure}[b]{0.192\linewidth}
        \includegraphics[width=\linewidth]{../img/5/quarry/worst/01-patch-3d-majavi-colormap-90.png}
        \caption{0.02cm}
        \label{fig : quarry-worst-10}
        \end{subfigure}
        \begin{subfigure}[b]{0.192\linewidth}
        \includegraphics[width=\linewidth]{../img/5/quarry/worst/02-patch-3d-majavi-colormap-100.png}
        \caption{0.02cm}
        \label{fig : quarry-worst-11}
        \end{subfigure}
        \begin{subfigure}[b]{0.192\linewidth}
        \includegraphics[width=\linewidth]{../img/5/quarry/worst/03-patch-3d-majavi-colormap-110.png}
        \caption{0.03cm}
        \label{fig : quarry-worst-12}
        \end{subfigure}
        \begin{subfigure}[b]{0.192\linewidth}
        \includegraphics[width=\linewidth]{../img/5/quarry/worst/04-patch-3d-majavi-colormap-120.png}
        \caption{0.05cm}
        \label{fig : quarry-worst-13}
        \end{subfigure}
        \begin{subfigure}[b]{0.192\linewidth}
        \includegraphics[width=\linewidth]{../img/5/quarry/worst/06-patch-3d-majavi-colormap-130.png}
        \caption{0.06cm}
        \label{fig : quarry-worst-14}
        \end{subfigure}
        \begin{subfigure}[b]{0.192\linewidth}
        \includegraphics[width=\linewidth]{../img/5/quarry/worst/07-patch-3d-majavi-colormap-140.png}
        \caption{0.08cm}
        \label{fig : quarry-worst-15}
        \end{subfigure}
        \begin{subfigure}[b]{0.192\linewidth}
        \includegraphics[width=\linewidth]{../img/5/quarry/worst/09-patch-3d-majavi-colormap-150.png}
        \caption{0.10cm}
        \label{fig : quarry-worst-16}
        \end{subfigure}
        \begin{subfigure}[b]{0.192\linewidth}
        \includegraphics[width=\linewidth]{../img/5/quarry/worst/11-patch-3d-majavi-colormap-160.png}
        \caption{0.11cm}
        \label{fig : quarry-worst-17}
        \end{subfigure}
        \begin{subfigure}[b]{0.192\linewidth}
        \includegraphics[width=\linewidth]{../img/5/quarry/worst/13-patch-3d-majavi-colormap-170.png}
        \caption{0.13cm}
        \label{fig : quarry-worst-18}
        \end{subfigure}
        \begin{subfigure}[b]{0.192\linewidth}
        \includegraphics[width=\linewidth]{../img/5/quarry/worst/15-patch-3d-majavi-colormap-180.png}
        \caption{0.15cm}
        \label{fig : quarry-worst-19}
        \end{subfigure}
        \begin{subfigure}[b]{0.192\linewidth}
        \includegraphics[width=\linewidth]{../img/5/quarry/worst/17-patch-3d-majavi-colormap-190.png}
        \caption{0.17cm}
        \label{fig : quarry-worst-20}
        \end{subfigure}
        \caption{Not traversable patches sampled from the test dataset correctly predicted by the model. We compute dthe Grad-CAM and applied as texture in the 3D render. The yellow highlights the ground region that contributed the most the model's prediction. }
        \label{fig : quarry-worst}
        \end{figure}
\subsection{False negative patches}
Not traversable inputs predicted as traversable by the model. Some of the grounds are similar to ones from the last section. Figure \ref{fig : quarry-worst-2} is almost identical to \ref{fig : quarry-false_negative-4}. In this case, the model must have misclassified the first part of the map thinking it could traverse till the two bumps in the end. If so, the robot should have been able to move for more than twenty centimeters. There are two grounds with a trail, \ref{fig : quarry-false_negative-14}, \ref{fig : quarry-false_negative-20}. In both cases, the model looked at the initial position of the robot, left, and the final part of the patch to understand if the robot fits on the trail. Moreover, there are different slopes in which Grad-CAM highlighted the region under the robot,  \ref{fig : quarry-false_negative-3},  \ref{fig : quarry-false_negative-4},  \ref{fig : quarry-false_negative-5},  \ref{fig : quarry-false_negative-7},  \ref{fig : quarry-false_negative-8},  \ref{fig : quarry-false_negative-19}. These regions may be hard to estimate when the sizes of the obstacles are closes to the edge cases. The model may overestimate when the size of the obstacle is just a few centimeters more than the real traversable one. Figure \ref{fig : quarry-false_negative-11} has a small step on the right region, correctly highlighted by Grad-CAM. Other patches have obstacles close to the end, \ref{fig : quarry-false_negative-10}, \ref{fig : quarry-false_negative-15}, \ref{fig : quarry-false_negative-16}, \ref{fig : quarry-false_negative-17}, \ref{fig : quarry-false_negative-18}. ONe patch clearly showed the model confusion, figure \ref{fig : quarry-false_negative-13}. Even if the obstacle was ahead, the model discriminated and an empty spot on the top left. 
in general, most of the region of interested on these patches are located on the left, close to the position of the robot's leg. Correctly, even if the prediction is wrong, the model looks at the first region of the surface, located near the legs, that can effect traversability. This shows the correct behavior even when misclassifying the inputs, meaning that the network is always looking in the correct spot. 
\begin{figure}[H]
    \centering
    \begin{subfigure}[b]{0.192\linewidth}
    \includegraphics[width=\linewidth]{../img/5/quarry/false_negative/-14-patch-3d-majavi-colormap-0.png}
    \caption{-0.14cm}
    \label{fig : quarry-false_negative-1}
    \end{subfigure}
    \begin{subfigure}[b]{0.192\linewidth}
    \includegraphics[width=\linewidth]{../img/5/quarry/false_negative/-6-patch-3d-majavi-colormap-10.png}
    \caption{-0.07cm}
    \label{fig : quarry-false_negative-2}
    \end{subfigure}
    \begin{subfigure}[b]{0.192\linewidth}
    \includegraphics[width=\linewidth]{../img/5/quarry/false_negative/-4-patch-3d-majavi-colormap-20.png}
    \caption{-0.05cm}
    \label{fig : quarry-false_negative-3}
    \end{subfigure}
    \begin{subfigure}[b]{0.192\linewidth}
    \includegraphics[width=\linewidth]{../img/5/quarry/false_negative/-2-patch-3d-majavi-colormap-30.png}
    \caption{-0.02cm}
    \label{fig : quarry-false_negative-4}
    \end{subfigure}
    \begin{subfigure}[b]{0.192\linewidth}
    \includegraphics[width=\linewidth]{../img/5/quarry/false_negative/00-patch-3d-majavi-colormap-40.png}
    \caption{-0.00cm}
    \label{fig : quarry-false_negative-5}
    \end{subfigure}
    \begin{subfigure}[b]{0.192\linewidth}
    \includegraphics[width=\linewidth]{../img/5/quarry/false_negative/01-patch-3d-majavi-colormap-50.png}
    \caption{0.01cm}
    \label{fig : quarry-false_negative-6}
    \end{subfigure}
    \begin{subfigure}[b]{0.192\linewidth}
    \includegraphics[width=\linewidth]{../img/5/quarry/false_negative/02-patch-3d-majavi-colormap-60.png}
    \caption{0.03cm}
    \label{fig : quarry-false_negative-7}
    \end{subfigure}
    \begin{subfigure}[b]{0.192\linewidth}
    \includegraphics[width=\linewidth]{../img/5/quarry/false_negative/04-patch-3d-majavi-colormap-70.png}
    \caption{0.04cm}
    \label{fig : quarry-false_negative-8}
    \end{subfigure}
    \begin{subfigure}[b]{0.192\linewidth}
    \includegraphics[width=\linewidth]{../img/5/quarry/false_negative/05-patch-3d-majavi-colormap-80.png}
    \caption{0.06cm}
    \label{fig : quarry-false_negative-9}
    \end{subfigure}
    \begin{subfigure}[b]{0.192\linewidth}
    \includegraphics[width=\linewidth]{../img/5/quarry/false_negative/06-patch-3d-majavi-colormap-90.png}
    \caption{0.07cm}
    \label{fig : quarry-false_negative-10}
    \end{subfigure}
    \begin{subfigure}[b]{0.192\linewidth}
    \includegraphics[width=\linewidth]{../img/5/quarry/false_negative/08-patch-3d-majavi-colormap-100.png}
    \caption{0.08cm}
    \label{fig : quarry-false_negative-11}
    \end{subfigure}
    \begin{subfigure}[b]{0.192\linewidth}
    \includegraphics[width=\linewidth]{../img/5/quarry/false_negative/09-patch-3d-majavi-colormap-110.png}
    \caption{0.09cm}
    \label{fig : quarry-false_negative-12}
    \end{subfigure}
    \begin{subfigure}[b]{0.192\linewidth}
    \includegraphics[width=\linewidth]{../img/5/quarry/false_negative/10-patch-3d-majavi-colormap-120.png}
    \caption{0.11cm}
    \label{fig : quarry-false_negative-13}
    \end{subfigure}
    \begin{subfigure}[b]{0.192\linewidth}
    \includegraphics[width=\linewidth]{../img/5/quarry/false_negative/11-patch-3d-majavi-colormap-130.png}
    \caption{0.12cm}
    \label{fig : quarry-false_negative-14}
    \end{subfigure}
    \begin{subfigure}[b]{0.192\linewidth}
    \includegraphics[width=\linewidth]{../img/5/quarry/false_negative/12-patch-3d-majavi-colormap-140.png}
    \caption{0.13cm}
    \label{fig : quarry-false_negative-15}
    \end{subfigure}
    \begin{subfigure}[b]{0.192\linewidth}
    \includegraphics[width=\linewidth]{../img/5/quarry/false_negative/14-patch-3d-majavi-colormap-150.png}
    \caption{0.14cm}
    \label{fig : quarry-false_negative-16}
    \end{subfigure}
    \begin{subfigure}[b]{0.192\linewidth}
    \includegraphics[width=\linewidth]{../img/5/quarry/false_negative/15-patch-3d-majavi-colormap-160.png}
    \caption{0.15cm}
    \label{fig : quarry-false_negative-17}
    \end{subfigure}
    \begin{subfigure}[b]{0.192\linewidth}
    \includegraphics[width=\linewidth]{../img/5/quarry/false_negative/16-patch-3d-majavi-colormap-170.png}
    \caption{0.17cm}
    \label{fig : quarry-false_negative-18}
    \end{subfigure}
    \begin{subfigure}[b]{0.192\linewidth}
    \includegraphics[width=\linewidth]{../img/5/quarry/false_negative/17-patch-3d-majavi-colormap-180.png}
    \caption{0.18cm}
    \label{fig : quarry-false_negative-19}
    \end{subfigure}
    \begin{subfigure}[b]{0.192\linewidth}
    \includegraphics[width=\linewidth]{../img/5/quarry/false_negative/18-patch-3d-majavi-colormap-190.png}
    \caption{0.18cm}
    \label{fig : quarry-false_negative-20}
    \end{subfigure}
    \label{fig : quarry-false_negative}
    \caption{Not traversable patches predicted as traversable sampled from the test dataset. We compute dthe Grad-CAM and applied as texture in the 3D render. The yellow highlights the ground region that contributed the most the model's prediction. }
    \end{figure}

\subsection{False positive patches}
False positive patches are the most interesting inputs. They are traversable patches classified as not. By looking at the features space, \ref{ fig :pca-test-patches}, we can noticed how the patches are mostly located close to the non traversable features. This explained why most of them are very close the surfaces presented before in figure \ref{fig : quarry-worst}. These samples included different types of terrains, some with obstacles ahead \ref{fig : false_positive-2}, \ref{fig : false_positive-3}, \ref{fig : false_positive-4}, \ref{fig : false_positive-7}, \ref{fig : false_positive-8}, \ref{fig : false_positive-11}, \ref{fig : false_positive-13}, \ref{fig : false_positive-14}, \ref{fig : false_positive-17}, \ref{fig : false_positive-19}, \ref{fig : false_positive-20}, slopes \ref{fig : false_positive-9}, \ref{fig : false_positive-10} and mixed grounds, \ref{fig : false_positive-1}, \ref{fig : false_positive-5}, \ref{fig : false_positive-12}, \ref{fig : false_positive-15}, \ref{fig : false_positive-16}, \ref{fig : false_positive-18}. In each case, the model identify meaningful features that can effect traversability. In the slopes, \ref{fig : false_positive-9} and \ref{false_positive-10}, the first part of the surface is highlighted. Similar to the traversable patches, the model tried to evaluated if the rear legs was able to move. This is also true for   \ref{fig : false_positive-6}. The only patch that consufed the model is the first one, \ref{fig : false_positive-18}, where it wrongly discriminated the flat region and not the obstacle. In all the others inputs, the network identifed importat region of the map that realistically could have caused the predicted not traversability.

\begin{figure}[H]
    \centering
    \begin{subfigure}[b]{0.192\linewidth}
    \includegraphics[width=\linewidth]{../img/5/quarry/false_positive/20-patch-3d-majavi-colormap-0.png}
    \caption{0.20cm}
    \label{fig : false_positive-1}
    \end{subfigure}
    \begin{subfigure}[b]{0.192\linewidth}
    \includegraphics[width=\linewidth]{../img/5/quarry/false_positive/20-patch-3d-majavi-colormap-12.png}
    \caption{0.21cm}
    \label{fig : false_positive-2}
    \end{subfigure}
    \begin{subfigure}[b]{0.192\linewidth}
    \includegraphics[width=\linewidth]{../img/5/quarry/false_positive/22-patch-3d-majavi-colormap-24.png}
    \caption{0.22cm}
    \label{fig : false_positive-3}
    \end{subfigure}
    \begin{subfigure}[b]{0.192\linewidth}
    \includegraphics[width=\linewidth]{../img/5/quarry/false_positive/23-patch-3d-majavi-colormap-36.png}
    \caption{0.23cm}
    \label{fig : false_positive-4}
    \end{subfigure}
    \begin{subfigure}[b]{0.192\linewidth}
    \includegraphics[width=\linewidth]{../img/5/quarry/false_positive/24-patch-3d-majavi-colormap-48.png}
    \caption{0.24cm}
    \label{fig : false_positive-5}
    \end{subfigure}
    \begin{subfigure}[b]{0.192\linewidth}
    \includegraphics[width=\linewidth]{../img/5/quarry/false_positive/25-patch-3d-majavi-colormap-60.png}
    \caption{0.25cm}
    \label{fig : false_positive-6}
    \end{subfigure}
    \begin{subfigure}[b]{0.192\linewidth}
    \includegraphics[width=\linewidth]{../img/5/quarry/false_positive/25-patch-3d-majavi-colormap-72.png}
    \caption{0.26cm}
    \label{fig : false_positive-7}
    \end{subfigure}
    \begin{subfigure}[b]{0.192\linewidth}
    \includegraphics[width=\linewidth]{../img/5/quarry/false_positive/26-patch-3d-majavi-colormap-84.png}
    \caption{0.27cm}
    \label{fig : false_positive-8}
    \end{subfigure}
    \begin{subfigure}[b]{0.192\linewidth}
    \includegraphics[width=\linewidth]{../img/5/quarry/false_positive/27-patch-3d-majavi-colormap-96.png}
    \caption{0.27cm}
    \label{fig : false_positive-9}
    \end{subfigure}
    \begin{subfigure}[b]{0.192\linewidth}
    \includegraphics[width=\linewidth]{../img/5/quarry/false_positive/27-patch-3d-majavi-colormap-108.png}
    \caption{0.28cm}
    \label{fig : false_positive-10}
    \end{subfigure}
    \begin{subfigure}[b]{0.192\linewidth}
    \includegraphics[width=\linewidth]{../img/5/quarry/false_positive/29-patch-3d-majavi-colormap-120.png}
    \caption{0.29cm}
    \label{fig : false_positive-11}
    \end{subfigure}
    \begin{subfigure}[b]{0.192\linewidth}
    \includegraphics[width=\linewidth]{../img/5/quarry/false_positive/30-patch-3d-majavi-colormap-132.png}
    \caption{0.30cm}
    \label{fig : false_positive-12}
    \end{subfigure}
    \begin{subfigure}[b]{0.192\linewidth}
    \includegraphics[width=\linewidth]{../img/5/quarry/false_positive/31-patch-3d-majavi-colormap-144.png}
    \caption{0.32cm}
    \label{fig : false_positive-13}
    \end{subfigure}
    \begin{subfigure}[b]{0.192\linewidth}
    \includegraphics[width=\linewidth]{../img/5/quarry/false_positive/33-patch-3d-majavi-colormap-156.png}
    \caption{0.34cm}
    \label{fig : false_positive-14}
    \end{subfigure}
    \begin{subfigure}[b]{0.192\linewidth}
    \includegraphics[width=\linewidth]{../img/5/quarry/false_positive/35-patch-3d-majavi-colormap-168.png}
    \caption{0.35cm}
    \label{fig : false_positive-15}
    \end{subfigure}
    \begin{subfigure}[b]{0.192\linewidth}
    \includegraphics[width=\linewidth]{../img/5/quarry/false_positive/36-patch-3d-majavi-colormap-180.png}
    \caption{0.36cm}
    \label{fig : false_positive-16}
    \end{subfigure}
    \begin{subfigure}[b]{0.192\linewidth}
    \includegraphics[width=\linewidth]{../img/5/quarry/false_positive/38-patch-3d-majavi-colormap-192.png}
    \caption{0.39cm}
    \label{fig : false_positive-17}
    \end{subfigure}
    \begin{subfigure}[b]{0.192\linewidth}
    \includegraphics[width=\linewidth]{../img/5/quarry/false_positive/41-patch-3d-majavi-colormap-204.png}
    \caption{0.41cm}
    \label{fig : false_positive-18}
    \end{subfigure}
    \begin{subfigure}[b]{0.192\linewidth}
    \includegraphics[width=\linewidth]{../img/5/quarry/false_positive/46-patch-3d-majavi-colormap-216.png}
    \caption{0.46cm}
    \label{fig : false_positive-19}
    \end{subfigure}
    \begin{subfigure}[b]{0.192\linewidth}
    \includegraphics[width=\linewidth]{../img/5/quarry/false_positive/51-patch-3d-majavi-colormap-228.png}
    \caption{0.52cm}
    \label{fig : false_positive-20}
    \end{subfigure}
    \label{fig : false_positive}
    \caption{Traversable patches predicted as not traversable sampled from the test dataset. We compute dthe Grad-CAM and applied as texture in the 3D render. The yellow highlights the ground region that contributed the most the model's prediction. }
    \end{figure}


% \end{document}
% \documentclass[../document.tex]{subfiles}
\begin{document}
\subsection{Robustness}
To test the model's robustness we createed custom patches with different features, walls/bumps/ramps, and test the model prediction against the real robot advancement obtained from the simulator. According to the previous experiments, we used a threshold of $20$cm and a time window of two seconds. 
\subsubsection{Wall in front of Krock}
The most trivial test is to place a not traversable wall in front of \emph{Krock} at an increasing distance from the its head. We will expect to reach a point where the model predict traversable even if the wall itself is too tall. Why? Because the robot will be able to travel more than the threshold before being stopped by the obstacle.

We created fifty-five different patches by first placing the wall exactly in front of the robot and then move it by $1$cm at the time towards the end. It follows some of the input patches ordered by distance from the robot. We remind to the reader that Krock traverse the patch from left to
 right.
\begin{figure}[H]
    \centering
    \begin{subfigure}[b]{0.24\textwidth}
    \includegraphics[width=\linewidth]{../img/5/custom_patches/walls_front/all/00-3d.png}
    \end{subfigure}
    % \begin{subfigure}[b]{0.24\textwidth}
    % \includegraphics[width=\linewidth]{../img/5/custom_patches/walls_front/all/50-3d.png}
    % \end{subfigure}
    \begin{subfigure}[b]{0.24\textwidth}
    \includegraphics[width=\linewidth]{../img/5/custom_patches/walls_front/all/04-3d.png}
    \end{subfigure}
    \begin{subfigure}[b]{0.24\textwidth}
    \includegraphics[width=\linewidth]{../img/5/custom_patches/walls_front/all/08-3d.png}
    \end{subfigure}
    \begin{subfigure}[b]{0.24\textwidth}
    \includegraphics[width=\linewidth]{../img/5/custom_patches/walls_front/all/12-3d.png}
    \end{subfigure}
    \begin{subfigure}[b]{0.24\textwidth}
    \includegraphics[width=\linewidth]{../img/5/custom_patches/walls_front/all/16-3d.png}
    \end{subfigure}
    \begin{subfigure}[b]{0.24\textwidth}
    \includegraphics[width=\linewidth]{../img/5/custom_patches/walls_front/all/18-3d.png}
    \end{subfigure}
    \begin{subfigure}[b]{0.24\textwidth}
    \includegraphics[width=\linewidth]{../img/5/custom_patches/walls_front/all/23-3d.png}
    \end{subfigure}
    % \begin{subfigure}[b]{0.24\textwidth}
    % \includegraphics[width=\linewidth]{../img/5/custom_patches/walls_front/all/15-3d.png}
    % \end{subfigure}
    \begin{subfigure}[b]{0.24\textwidth}
    \includegraphics[width=\linewidth]{../img/5/custom_patches/walls_front/all/25-3d.png}
    \end{subfigure}
    \caption{Some of the tested patches with a non traversable wall at increasing distance from Krock.}
    \end{figure}

Given those inputs to the model, we get the following predictions.
\begin{figure}[H]
    \centering
\begin{subfigure}[b]{1\textwidth}
    \includegraphics[width=\linewidth]{../img/5/custom_patches/walls_front/predictions.png}
    \end{subfigure}
    \caption{Traversability probabilities against wall distance from Krock's head.}
\end{figure}
\todo[inline]{graph too tall, adjust the figure size}
Summarized by the following table:
\begin{table}[H]
    \centering
    \begin{tabular}{l|cc}
        Distance(cm) & Prediction \\ 
        \hline
        0 - 20  & Not traversable \\ 
        20 - end & Traversable \\ 
        \hline
    \end{tabular}
    \caption{Model prediction from the wall patches.}
\end{table}
To be sure the results are correct, we run the last not traversable  and the first traversable patch on the simulator to get real advancement. In the simulator, Krock advances $19.9$cm on the not traversable patch $(a)$ where the wall is at a distance of $19.6$cm from the head. While, on the first traversable patch in which the wall is at a distance of $20.5$cm, the robot was able to travel for $22.4$cm. This shows the network ability to correct understanding that distance from the obstacle is more relevant than its height.
\begin{figure}[H]
    \centering
    \begin{subfigure}[b]{0.33\textwidth}
        \includegraphics[width=\linewidth]{../img/5/custom_patches/walls_front/1-2d.png}
        \end{subfigure}   
    \begin{subfigure}[b]{0.33\textwidth}
        \includegraphics[width=\linewidth]{../img/5/custom_patches/walls_front/2-2d.png}
    \end{subfigure}   \\
    \begin{subfigure}[b]{0.33\textwidth}
        \includegraphics[width=\linewidth]{../img/5/custom_patches/walls_front/1-3d.png}
        \caption{Distance  $20$cm}
    \end{subfigure}   
    \begin{subfigure}[b]{0.33\textwidth}
        \includegraphics[width=\linewidth]{../img/5/custom_patches/walls_front/1-3d.png}
        \caption{Distance $22$cm}
    \end{subfigure}   
    \caption{The last non traversable and the first traversable patches with wall at a distance $ < tr$ and $ > $ tr.}
\end{figure}
Due to the patch resolution and minor changes in the spawing position, the robot may advance more than the distance between its head and the wall. 

Furthermore, we can increase the wall size of the first traversable patch, $(b)$, to $10$ and to $100$m to stress even more the ability of the model to look at the distance and not at the height.

\begin{figure}[H]
    \centering
    \begin{subfigure}[b]{0.33\textwidth}
        \includegraphics[width=\linewidth]{../img/5/custom_patches/walls_front/big-1-2d.png}
    \caption{height $=10$m}
    \end{subfigure}   
    \begin{subfigure}[b]{0.33\textwidth}
        \includegraphics[width=\linewidth]{../img/5/custom_patches/walls_front/big-2-2d.png}
        \caption{height $=100$m}
    \end{subfigure}   
\caption{Two patches with a very tall wall at a distance $> tr$.}    
\end{figure}
\todo[inline]{remove title, second colormap is wrong}
Correctly, the model classifies the patches as traversable and was not confused by the enourmous height of the wall.

\subsubsection{Increasing height walls}
We can also evaluate the model's robustness by placing different wall  of increasing heights in front of the robot to check whether the prediction matches the real data. We run forty patches in the simulator from a wall's height of $1$cm to $20$cm. The following figure shows some of the inputs.

\begin{figure}[H]
    \centering
    \begin{subfigure}[b]{0.24\textwidth}
    \includegraphics[width=\linewidth]{../img/5/custom_patches/walls_increasing/all/00-3d.png}
    \end{subfigure}
    \begin{subfigure}[b]{0.24\textwidth}
    \includegraphics[width=\linewidth]{../img/5/custom_patches/walls_increasing/all/03-3d.png}
    \end{subfigure}
    \begin{subfigure}[b]{0.24\textwidth}
    \includegraphics[width=\linewidth]{../img/5/custom_patches/walls_increasing/all/06-3d.png}
    \end{subfigure}
    \begin{subfigure}[b]{0.24\textwidth}
    \includegraphics[width=\linewidth]{../img/5/custom_patches/walls_increasing/all/09-3d.png}
    \end{subfigure}
    \begin{subfigure}[b]{0.24\textwidth}
    \includegraphics[width=\linewidth]{../img/5/custom_patches/walls_increasing/all/11-3d.png}
    \end{subfigure}
    \begin{subfigure}[b]{0.24\textwidth}
    \includegraphics[width=\linewidth]{../img/5/custom_patches/walls_increasing/all/14-3d.png}
    \end{subfigure}
    \begin{subfigure}[b]{0.24\textwidth}
    \includegraphics[width=\linewidth]{../img/5/custom_patches/walls_increasing/all/17-3d.png}
    \end{subfigure}
    \begin{subfigure}[b]{0.24\textwidth}
    \includegraphics[width=\linewidth]{../img/5/custom_patches/walls_increasing/all/19-3d.png}
    \end{subfigure}
    \caption{Some of the tested patches with a wall at increasing height ahead of Krock.}
    \end{figure}
The models predicts that the walls under $10$cm are traversable.
\begin{figure}[H]
    \centering
\begin{subfigure}[b]{1\textwidth}
    \includegraphics[width=\linewidth]{../img/5/custom_patches/walls_increasing/predictions.png}
    \end{subfigure}
    \caption{Traversability probabilities against walls height in front of Krock.}
\end{figure}

\begin{table}[H]
    \centering
    \begin{tabular}{l|cc}
        Height(cm) & Prediction \\ 
        \hline
        0 - 10  & Traversable \\ 
        10 - end & Not Traversable \\ 
        \hline
    \end{tabular}
    \caption{Model prediction for the wall patches}
\end{table}
We can compare the model's prediction with the advancement computed in the simulator using the same approach from the last section. The following figure shows the results from the last traversable patch and the first non traversable.

\begin{figure}[H]
    \centering
    \begin{subfigure}[b]{0.33\textwidth}
        \includegraphics[width=\linewidth]{../img/5/custom_patches/walls_increasing/1-2d.png}
    \end{subfigure}   
    \begin{subfigure}[b]{0.33\textwidth}
        \includegraphics[width=\linewidth]{../img/5/custom_patches/walls_increasing/2-2d}
    \end{subfigure}   
    \begin{subfigure}[b]{0.33\textwidth}
        \includegraphics[width=\linewidth]{../img/5/custom_patches/walls_increasing/1-3d.png}
    \caption{height $9.7$cm}
    \end{subfigure}   
    \begin{subfigure}[b]{0.33\textwidth}
        \includegraphics[width=\linewidth]{../img/5/custom_patches/walls_increasing/2-3d}
        \caption{Height $10$cm}
    \end{subfigure}   
\caption{The last traversable and the first non traversable patches   with a increasing height wall ahead of Krock.}    
\end{figure}
In the first case, the simulator outputs and advancement of $39.5$cm meaning that Krock was able to overcome the obstacle, while it fails in the second case. Correctly, the predictions match the real data.

\subsubsection{Increasing height and distance walls}
We can combine the previous experiments and test the model prediction against the ground truth for each height/distance combination. To reduce the number of samples and improve readability, we limit ourself to consider only patches with a wall tall between $5$cm and $20$cm, we know from previous sections patches with a value smaller and bigger obstacle are traversable and not traversable respectively. Similar, we set the wall's distance from Krock's head between $1$cm to $30$cm for the same reasons. The following image shows the traversability probability for each patch.
\begin{figure}[H]
    \centering
\begin{subfigure}[b]{1\textwidth}
    \includegraphics[width=\linewidth]{../img/5/custom_patches/walls_heights/walls_heights.png}
    \end{subfigure}
    \caption{Traversability probabilities for patches with a wall of increasing height and distance form Krock's head.}
\end{figure}


\todo[inline]{missing ground truth heatmap}
\todo[inline]{FINISH}
% \subsubsection{Wall under Krock}
% By placing a wall taller than the Krock's distance from the ground the patch won't be traversable anymore since the robot will be lifted up or the back leg wil be stocked. So, as we did in the previous section we created forty patches with a wall from $1$cm to $20$cm in front of the rear legs.

% \begin{figure}[H]
%     \centering
%     \begin{subfigure}[b]{0.160\textwidth}
%     \includegraphics[width=\linewidth]{../img/5/custom_patches/wall_under/all/00-3d.png}
%     \end{subfigure}
%     \begin{subfigure}[b]{0.160\textwidth}
%     \includegraphics[width=\linewidth]{../img/5/custom_patches/wall_under/all/05-3d.png}
%     \end{subfigure}
%     \begin{subfigure}[b]{0.160\textwidth}
%     \includegraphics[width=\linewidth]{../img/5/custom_patches/wall_under/all/10-3d.png}
%     \end{subfigure}
%     \begin{subfigure}[b]{0.160\textwidth}
%     \includegraphics[width=\linewidth]{../img/5/custom_patches/wall_under/all/15-3d.png}
%     \end{subfigure}
%     \begin{subfigure}[b]{0.160\textwidth}
%     \includegraphics[width=\linewidth]{../img/5/custom_patches/wall_under/all/20-3d.png}
%     \end{subfigure}
%     \begin{subfigure}[b]{0.160\textwidth}
%     \includegraphics[width=\linewidth]{../img/5/custom_patches/wall_under/all/25-3d.png}
%     \end{subfigure}
%     \begin{subfigure}[b]{0.160\textwidth}
%     \includegraphics[width=\linewidth]{../img/5/custom_patches/wall_under/all/30-3d.png}
%     \end{subfigure}
%     \begin{subfigure}[b]{0.160\textwidth}
%     \includegraphics[width=\linewidth]{../img/5/custom_patches/wall_under/all/35-3d.png}
%     \end{subfigure}
%     \caption{Wall under Krock}
%     \end{figure}
    
% The following table summarizes the results
% \begin{table}[H]
%     \centering
%     \begin{tabular}{l|cc}
%         Height(cm) & Prediction \\ 
%         \hline
%         0 - 5cm  & Traversable \\ 
%         5 - end & Not Traversable \\ 
%         \hline
%     \end{tabular}
%     \caption{Model prediction from the wall patches}
% \end{table}
    
\subsubsection{Tunnel}
\todo[inline]{do it}

\begin{figure}[H]
    \centering
    \begin{subfigure}[b]{0.24\textwidth}
    \includegraphics[width=\linewidth]{../img/5/custom_patches/tunnel/all/00-3d.png}
    \end{subfigure}
    \begin{subfigure}[b]{0.24\textwidth}
    \includegraphics[width=\linewidth]{../img/5/custom_patches/tunnel/all/04-3d.png}
    \end{subfigure}
    \begin{subfigure}[b]{0.24\textwidth}
    \includegraphics[width=\linewidth]{../img/5/custom_patches/tunnel/all/08-3d.png}
    \end{subfigure}
    \begin{subfigure}[b]{0.24\textwidth}
    \includegraphics[width=\linewidth]{../img/5/custom_patches/tunnel/all/11-3d.png}
    \end{subfigure}
    \begin{subfigure}[b]{0.24\textwidth}
    \includegraphics[width=\linewidth]{../img/5/custom_patches/tunnel/all/13-3d.png}
    \end{subfigure}
    \begin{subfigure}[b]{0.24\textwidth}
    \includegraphics[width=\linewidth]{../img/5/custom_patches/tunnel/all/16-3d.png}
    \end{subfigure}
    \begin{subfigure}[b]{0.24\textwidth}
    \includegraphics[width=\linewidth]{../img/5/custom_patches/tunnel/all/20-3d.png}
    \end{subfigure}
    \begin{subfigure}[b]{0.24\textwidth}
    \includegraphics[width=\linewidth]{../img/5/custom_patches/tunnel/all/23-3d.png}
    \end{subfigure}
    \caption{Some of the tested patches with tunnel at different distances.}
    \end{figure}

\subsubsection{Ramps}
\todo[inline]{explain we had to square the linear ramps to create a small flat region}
We generate twenty ramps with a maximum height from $0.25$m to $4$m. Below we plot the traversability probabilities against the maximum height of each ramp.

\begin{figure}[H]
    \centering
    \begin{subfigure}[b]{0.24\textwidth}
    \includegraphics[width=\linewidth]{../img/5/custom_patches/ramp/all/00-3d.png}
    \end{subfigure}
    \begin{subfigure}[b]{0.24\textwidth}
    \includegraphics[width=\linewidth]{../img/5/custom_patches/ramp/all/03-3d.png}
    \end{subfigure}
    \begin{subfigure}[b]{0.24\textwidth}
    \includegraphics[width=\linewidth]{../img/5/custom_patches/ramp/all/06-3d.png}
    \end{subfigure}
    \begin{subfigure}[b]{0.24\textwidth}
    \includegraphics[width=\linewidth]{../img/5/custom_patches/ramp/all/09-3d.png}
    \end{subfigure}
    \begin{subfigure}[b]{0.24\textwidth}
    \includegraphics[width=\linewidth]{../img/5/custom_patches/ramp/all/11-3d.png}
    \end{subfigure}
    \begin{subfigure}[b]{0.24\textwidth}
    \includegraphics[width=\linewidth]{../img/5/custom_patches/ramp/all/14-3d.png}
    \end{subfigure}
    \begin{subfigure}[b]{0.24\textwidth}
    \includegraphics[width=\linewidth]{../img/5/custom_patches/ramp/all/17-3d.png}
    \end{subfigure}
    \begin{subfigure}[b]{0.24\textwidth}
    \includegraphics[width=\linewidth]{../img/5/custom_patches/ramp/all/19-3d.png}
    \end{subfigure}
    \caption{Some of the tested patches with steep ramps.}
    \end{figure}

\begin{figure}[H]
    \centering
\begin{subfigure}[b]{1\textwidth}
    \includegraphics[width=\linewidth]{../img/5/custom_patches/ramp/predictions.png}
    \end{subfigure}
    \caption{Traversability probabilities against maximum height of each ramp.}
\end{figure}
\todo[inline]{x labels are wrong, why?}
The following table summarizes the results.

\begin{table}[H]
    \centering
    \begin{tabular}{l|cc}
        Height(m) & Prediction \\ 
        \hline
        0.5 - 1  &  Traversable \\ 
        1 - end & Not traversable \\ 
        \hline
    \end{tabular}
    \caption{Model prediction for the ramps patches}
\end{table}
We test the last traversable patch and the first not traversable with the real advancement gather from the simulator.

\begin{figure}[H]
    \centering
    \begin{subfigure}[b]{0.33\textwidth}
        \includegraphics[width=\linewidth]{../img/5/custom_patches/ramp/ramp-6-2d.png}
    \end{subfigure}   
    \begin{subfigure}[b]{0.33\textwidth}
        \includegraphics[width=\linewidth]{../img/5/custom_patches/ramp/ramp-7-2d}
    \end{subfigure}   
    \begin{subfigure}[b]{0.33\textwidth}
        \includegraphics[width=\linewidth]{../img/5/custom_patches/ramp/ramp-6-3d.png}
    \caption{Height $0.94$cm}
    \end{subfigure}   
    \begin{subfigure}[b]{0.33\textwidth}
        \includegraphics[width=\linewidth]{../img/5/custom_patches/ramp/ramp-7-3d}
        \caption{height $1.1$m}
    \end{subfigure}   
\caption{The last traversable and the first non traversable patches with a steep ramp ahead of Krock.}    
\end{figure}
\todo[inline]{scale is wrong}
Krock is able to traverse up to $1m$ height ramps, this is confirmed using the simulation.

We can add rocks to those patches to give Krock the ability to climb them better. 
\todo[inline]{add rocks}

\subsubsection{Holes}
\todo[inline]{do it}
\end{document}
%

\end{document}